%%%%%%%%%%%%%%%%%%%%%%%%%%%%%%%%%%%%%%%%
% Classe do documento
%%%%%%%%%%%%%%%%%%%%%%%%%%%%%%%%%%%%%%%%

% Nós usamos a classe "unb-cic".  Deixe apenas uma das linhas
% abaixo não-comentada, dependendo se você for do bacharelado ou
% da licenciatura.

\documentclass[bacharelado]{unb-cic}
%\documentclass[licenciatura]{unb-cic}



%%%%%%%%%%%%%%%%%%%%%%%%%%%%%%%%%%%%%%%%
% Pacotes importados
%%%%%%%%%%%%%%%%%%%%%%%%%%%%%%%%%%%%%%%%

\usepackage[brazil,american]{babel}
\usepackage[T1]{fontenc}
\usepackage{indentfirst}
\usepackage{natbib}
\usepackage{xcolor,graphicx,url}
\usepackage{amsmath,amssymb,amsthm}		%Pacote da AMS para usar fórmulas
\usepackage[utf8]{inputenc}
\usepackage{listings}					%Formatação do código fonte
\usepackage{textcomp}
\usepackage{enumerate}					%Enumeração de listas
\usepackage{rotating}
\usepackage{usecases}

%%%%%%%%%%%%%%%%%%%%%%%%%%%%%%%%%%%%%%%%
% Cores dos links
%%%%%%%%%%%%%%%%%%%%%%%%%%%%%%%%%%%%%%%%

% Veja o arquivos cores.tex se quiser ver que outras cores estão
% pré-definidas.  Utilizando o comando \hypersetup abaixo nós
% evitamos aquelas caixas vermelhas feias em volta dos links.

\input{cores}
\hypersetup{
  colorlinks=true,
  linkcolor=DarkScarletRed,
  citecolor=DarkScarletRed,
  filecolor=DarkScarletRed,
  urlcolor= DarkScarletRed
}


\lstset{literate=%
	{é}{{\'{e}}}1
	{ç}{{\'{c}}}1
	{ã}{{\'{c}}}1
	{è}{{\`{e}}}1
    {ê}{{\^{e}}}1
    {û}{{\^{u}}}1
    {ù}{{\`{u}}}1
    {â}{{\^{a}}}1
    {à}{{\`{a}}}1
    {ç}{{\c{c}}}1
    {Ç}{{\c{C}}}1
    {É}{{\'{E}}}1
    {Ê}{{\^{E}}}1
    {À}{{\`{A}}}1
    {Â}{{\^{A}}}1
    {Î}{{\^{I}}}1
    {î}{{\^{i}}}1
    {í}{{\'{i}}}1
    {á}{{\'{a}}}1
    {ã}{{\~{a}}}1
    {Ã}{{\~{A}}}1
    {õ}{{\~{o}}}1
    {Õ}{{\~{O}}}1
    {ó}{{\'{o}}}1
}

%%%%%%%%%%%%%%%%%%%%%%%%%%%%%%%%%%%%%%%%
% Informações sobre a monografia
%%%%%%%%%%%%%%%%%%%%%%%%%%%%%%%%%%%%%%%%

\title{Estudo, Definição e Implementação de ... Baseado em Modelo Multidimesional e Abordagem de Sistema Multiagente}

\orientador{\prof Célia Ralha}{CIC/UnB}
\coordenador{\prof Flávio de Barros Vidal}{CIC/UnB}
\diamesano{12}{Março}{2013}

\membrobanca{\prof Germana Nóbrega de Menezes}{CIC/UnB}
\membrobanca{\prof Fernanda Lima}{CIC/UnB}

\autor{João Paulo de Freitas}{Matos}
\CDU{004.4}

\palavraschave{Sistemas Multiagentes, Informática na Educação, Modelo Multidimensional}
\keywords{Multiagent Systems, Informatics in Education, Multidimensional Model}

%%%%%%%%%%%%%%%%%%%%%%%%%%%%%%%%%%%%%%%%
% Texto
%%%%%%%%%%%%%%%%%%%%%%%%%%%%%%%%%%%%%%%%

\begin{document}
  \maketitle
  \pretextual

  \begin{dedicatoria}
  Dedico a....
  \end{dedicatoria}

  \begin{agradecimentos}
  Agradeço a....
  \end{agradecimentos}

  \begin{resumo}
  A ciência...
  \end{resumo}

  \selectlanguage{american}
  \begin{abstract}
  The science...
  \end{abstract}
  \selectlanguage{brazil}

  \tableofcontents
  \listoffigures
  \listoftables

  \textual
  \chapter{Introdução}

Cada pessoa possui uma forma preferencial de absorção do conhecimento. Seja por imagens, textos, teoria ou prática, durante uma situação de aprendizagem todos tendem a receber e processar melhor as informações que são recebidas de certa maneira, em detrimento a outras. Essa forma de recepção do conhecimento é nomeada estilo de aprendizagem.
 
O acesso à recursos tecnológicos, outrora caros e difíceis, tornaram-se presentes no cotidiano de muitas pessoas devido a facilidade de aquisição. Dessa forma, a viabilidade do aprendizado por meio do computador aumentou e começa a modificar o paradigma do professor detentor do conhecimento e o único responśavel pela sua transmissão.

Em um ambiente escolar a forma didática escolhida por um docente pode afetar o desempenho dos seus alunos, visto que a forma de transmissão do conhecimento escolhida pode prejudicar alguns estudantes. A situação ideal requer a existência de uma personalização do ensino para cada estudante, considerando as características inerentes à sua cognição. Inspirado por esse ideal, algumas ferramentas no âmbito da computação pretendem o ensino personalizado aos estudantes de acordo com o seu perfil de utilização.

Assim, conhecer os fatores relacionados ao processo de aprendizagem exige que as ferramentas computacionais aplicadas ao ensino consigam determinar eficientemente os estilos de aprendizagem. Porém, a complexidade das aplicações aumenta devido às abordagens que serão empregadas, bem como o aumento da exigência de processamento.

Algumas abordagens visam diminuir essa complexidade, permitindo a representação aproximada das características dos alunos em ambientes computacionais, de forma que seja possível representar diversos domínios de modelo para um único estudante. Por exemplo, o modelo do estudante elaborado na forma multidimensional a partir dos universos cognitivo, metacognitivo e afetivo.

Somado à isso, técnicas de computação distribuída são projetadas para rodar de forma descentralizada, com componentes e serviços rodando em diversos lugares distintos e comunicando-se uma com as outras através de mensagens. A arquitetura dessas aplicações é projetada objetivando o alto paralelismo, exibilidade, interoperabilidade, dentre outros aspectos, permitindo a coexistência de diversos modelos de estudantes em uma única aplicação.

Portanto, determinar o estilo de aprendizagem mostra-se uma estratégia fundamental para que a transmissão de informações e vivências entre alunos e professores torne-se mais eficaz e perceptível, promovendo a criação de informações cada vez mais relevantes para o planejamento, acompanhamento e avaliação dos aprendizes.

\section{Problema}
Os ambientes educacionais de aprendizagem não possuem uma arquitetura apropriada para a inferência de modelos multidimensionais. A análise do modelo multidimensional do aluno identificam suas características particulares, compreendendo os modelos cognitivo, metacognitivo e afetivo, tornando-se informações fundamentais para o docente na busca pela didática mais recomendável para seus alunos. Porém a arquitetura cliente-servidor permite uma abordagem simples do problema, não usufruindo da possibilidade de representação do conhecimento individualizado em tempo real para a interação entre modelos de estudantes.

\section{Objetivos}
Tendo em vista o cenário atual apresentado, o presente trabalho tem como objetivo definir uma arquitetura distribuída na~\emph{web} para auxiliar o processo de ensino-aprendizagem por meio da abordagem de Sistema multiagente (SMA), criando insumos para auxiliar o docente na adoção da melhor estratégia didática de ensino.

Esta abordagem permitirá a construção e manutenção de um modelo multidimensional do estudante, a partir do qual os estilos de aprendizagem desse estudante poderão ser identificados. A abordagem de sistemas multiagentes permite a decomposição do problema na modelagem multidimensional em vários subproblemas menores, sendo capaz de diminuir a complexidade da resolucão. Além disso as características inerentes aos agentes, por exemplo a habilidade social, podem permitir a interação com outros modelos de alunos visando comparações e validações do modelo.

Em corroboração ao objetivo do trabalho, a informação do estilo de aprendizagem ao docente pode permitir a correta orientação da sua didática em sala de aula, melhorando qualitativamente o ensino aos seus alunos.

Especificamente, os objetivos deste trabalho são:
\begin{itemize}
 	\item Projeto da arquitetura geral de um SMA, utilizando-se de uma metodologia apropriada;
	\item Definir e implementar a arquitetura geral da solução, incluindo os agentes assistentes de cognição, metacognição e afetivo;
	\item Propor uma interface do agente assistente de cognição com os atores externos do sistema: Docente e Estudante;
\end{itemize}

\section{Metodologia}

A metodologia de realização deste trabalho é constituída da revisão bibliográfica, modelagem da arquitetura, implementação e testes em laboratório da solução. A primeira etapa consistiu do estudo dos conceitos de Informática na Educação (IE), Sistemas Multiagentes e levantamentos de uso de SMA em contextos pegagógicos. Este estudo foi importante para a orientação do desenvolvimento embasado na teoria e a comparação das ferramentas existentes com a solução proposta.

Em seguida, foi realizado o levantamento e estudo de metodologias aproriadas para modelagem de SMA existentes para a modelagem e construção da solução. Essas metodologias foram comparadas e uma delas escolhida para a modelagem do SMA.

O desenvolvimento da aplicação multiagente foi feito com o~\emph{framework}~\emph{JADE} o qual é completamente desenvolvido na linguagem~\emph{JAVA} e simplifica a implementação de SMA que cumprem as especificações FIPA. 

Em seguida, foi necessário a escolha de uma ferramenta para desenvolvimento da interface~\emph{web} e a sua integração com o SMA. Baseado nos estudos feitos, o~\emph{framework JBoss Seam} mostrou-se mais viável visto que facilita a construção de aplicações dinâmicas para a internet de forma simples e ágil.

Por fim, foram realizados testes em laboratório da solução proposta. Foram definidos cenários que simulavam o uso da aplicação por meio dos perfis de Aluno e Docente.

\section{Estrutura do Trabalho}
A divisão de capítulos foi feita da seguinte forma:
\begin{itemize}
	\item Capítulo 2: Apresenta uma breve visão sobre as áreas de estudo envolvidas neste trabalho, tais como IE e SMA. Além disso, apresenta algumas ferramentas e tecnologias utilizadas, bem como os trabalhos correlatos e um comparativo.
	\item Capítulo 3: Contém a proposta de solução composta pela metodologia, modelagem da arquitetura e implementação.
	\item Capítulo 4: Exibe os testes realizados em laboratório por meio dos perfis dos usuários: Aluno e Docente.
	\item Capítulo 5: Apresenta algumas conclusões e abre perspectivas para trabalhos futuros.
	\item Capítulo 6: Apresenta o apêndice com todos os casos de uso escritos na modelagem do SMA.
\end{itemize}

  \chapter{Fundamentos básicos}

Este capitulo apresenta os principais conceitos e definições necessários para o entendimento deste trabalho. A seção 2.1 apresenta definições de sistemas multiagentes.
A seção 2.2 apresenta uma breve explicação sobre a arquitetura da plataforma JAMA, bem como o seu funcionamento.
A seção 2.3 contém uma breve introdução sobre dependabilidade.
Por fim, a seção 2.4 tem um texto expondo o funcionamento do JAMA, bem como a sua arquitetura.

\section{Sistemas Multiagentes}

\subsection{Inteligência artificial}

Antes de explicarmos o conceito de sistemas multiagentes (SMA), é necessário mostrar conceitos que são base para o entendimento de SMA. Inicia-se apresentando alguns conceitos de Inteligencia Artificial (IA). De acordo com~\cite{poole98} identificamos que a definição de IA pode variar em duas dimensões principais. Usando a definição de sistemas computacionais que agem racionalmente temos:

\begin{quote}
\emph{Computational Intelligence is the study of the design of intelligent agents.}
\end{quote}

Nessa definição, é importante ressaltar que o agente é uma entidade que atua racionalmente, esperando-se que essa racionalidade e outras características o diferencie de simples programas.

Com o crescimento dos estudos relacionado a este campo, a inteligência artificial ganhou várias áreas de atuação e resolução de problemas no nosso cotidiano. Um dos problemas é a necessidade de executar aplicações que resolvem problemas de alta complexidade. Essas aplicações podem exigir um hardware muito caro para a execução, ou então pode-se usar a abordagem de distribuí-la em vários computadores que dividem a sua execução. É justamente onde entra a inteligência artificial distribuída: São sistemas que são compostos por vários agentes coletivos, ou seja, distribuem o trabalho uns com os outros. Cada agente pode possuir uma capacidade diferente, sendo possível realizar a tarefa de modo paralelo. 

\subsection{Agente}

De acordo com~\cite{novig95}, agentes são entidades (reais ou virtuais) que funcionam de forma autônoma em um ambiente, ou seja, não necessitam de intervenção humana para realizar processamento. Esse ambiente de funcionamento do agente geralmente contém vários outros agentes e é possível a comunicação entre eles através do ambiente por meio de troca de mensagens.

Em geral o funcionamento de agentes acontece de forma a perceberem o ambiente em que estão por meio de sensores, fazem análises com base nessa interação inicial e por fim podem agir sobre o ambiente de forma a modifica-lo por meio de efetuadores. A figura~\ref{fig:agente-basico} apresenta um resumo do que foi dito.

\begin{figure}
	\centering
	\includegraphics[scale=0.75]{images/agente-basico.png}
	\caption{Esquematização do funcionamento básico de um agente em um ambiente.}
	\label{fig:agente-basico}
\end{figure}

Agentes racionais seguem o princípio de racionalidade básico: sempre objetivam suas ações pela escolha da melhor ação possível segundo seus conhecimentos. Logo é possível inferir que a ação de um agente nem sempre alcança o máximo desempenho, sendo desempenho o parâmetro definido para medir o grau de sucesso da ação de um agente com base nos seus objetivos.

Como dito anteriormente, agentes estão presentes em um ambiente. O agente não tem controle total do ambiente, ele pode no máximo influenciá-lo com a sua atuação. Podemos separar ambientes em classes: Software, Físico e Relidade virtual (simulação de ambientes reais em software). De acordo com~\cite{wooldridge04} temos, em geral, ambientes tem propriedades inerentes que dizem respeito ao seu funcionamento:

\begin{itemize}
	\item Observável: Neste tipo de ambiente, os sensores dos agentes conseguem ter percepção completa do ambiente. Por exemplo, um sensor de movimento consegue ter visão total em um ambiente aberto.
	\item Determinística: O próximo estado do ambiente é sempre conhecido dado o estado atual do ambiente e as ações dos agentes. O oposto do ambiente determinístico é o estocástico, quando não temos certeza do estado do ambiente. Por exemplo, agentes dependentes de eventos climáticos.
	\item Episódico: A experiência do agente é dividida em episódios, onde cada episódio é a percepção do agente e a sua ação.
	\item Sequêncial: A ação tomada pelo agente pode afetar o estado do ambiente e ocasionar na mudança de estado
	\item Estático: O ambiente não é alterado enquanto um agente escolhe uma ação.
	\item Discreto: Existe um número definido de ações e percepções do agente para o ambiente em cada turno.
	\item Contínuo: As percepções e ações de um agente modificam-se em um espectro contínuo de valores. Por exemplo, temperatura de um sensor muda de forma contínua.
\end{itemize}

Na tabela~\ref{lista_agentes} mostramos alguns exemplos de agentes, apresentando as suas características já discutidas nesse trabalho.

\begin{table}
	\caption{Listagem de sistemas multiagentes com propriedades de medida de performance, ambiente, atuadores e sensores}
	\begin{tabular}{|p{3cm} | p{3cm} | p{2cm}| p{3cm} | p{3cm} |}
		\hline
		\textbf{Tipo de agente}	& \textbf{Medida de performance} & \textbf{Ambiente} & \textbf{Atuadores}  & \textbf{Sensores}	\\
		\hline
		Sensores de estacionamento	& Avarias no veículo & Carro e garagens & Freio do carro, controle de velocidade & Sensor de proximidade	\\
		\hline
		Jogos com oponente computador	& Quantidade de vitórias &	Software & Realizar jogada & Percepção do tabuleiro	\\
		\hline
		Agentes hospitalares		& Saúde do paciente & Paciente, ambiente médico & Diagnósticos & Entrada de sintomas do paciente	\\
		\hline
	\end{tabular}
	\label{lista_agentes}
\end{table}
 
A primeira linha da tabela~\ref{lista_agentes} é apresentado um exemplo de um agente atuando em um veículo como um sensor de estacionamento. Responsável por auxiliar o motorista no ato de estacionar o carro, o seu ambiente é da classe físico (considerando o carro e o ambiente onde está o carro). Seu sensor de proximidade é a percepção do ambiente e caso detecte que está próximo de um obstáculo pode atuar nos freios dos carros diminuindo a velocidade e evitando colisões. Avarias no carro podem indicar um mal funcionamento do sensor.

A segunda linha da tabela é apresentado u exemplo de agente atuando em um jogo qualquer. Esse ambiente é dito dinâmico, pois a cada jogada de um oponente (real ou não), o agente irá analisar a jogada feita pelo seu oponente, irá calcular sua próxima jogada e irá realizá-la. O objetivo principal do agente é a vitória. O ambiente que o agente atua é um software e o seu atuador é um algum mecanismo que permite que ele realize a jogada. O sensor é o mecanismo no qual o agente irá perceber a jogada realizada pelo oponente.

Por fim, última linha da tabela~\ref{lista_agentes} expõe um exemplo de um agente médico atuando em um ambiente estático um paciente. Esse ambiente é dito estático por que não será alterado pelo agente nesse exemplo, mas podendo ser diferente dependendo da aplicação. O objetivo principal é monitorar a saúde do paciênte, logo a medida de performance será a aproximação ou não do diagnóstico médico. Seu atuador não será diretamente no ambiente (corpo humano), será na forma de relatórios médicos e seus sensores podem variar de acordo com a doença a ser monitorada.

Conforme podemos encontrar em~\cite{wooldridge04}, podemos definir algumas noções gerais de agentes. A primeira, chamada de noção fraca, contém a maior parte dos agentes. Ela compreende os aspectos de \emph{reatividade}, \emph{proatividade} e \emph{habilidade social}. O conceito de reatividade  está ligado com o agente perceber o ambiente e reagir. Proatividade é a característica do agente tomar a iniciativa e agir sem a necessidade de nenhum estímulo. Habilidade social é a capacidade de interação com outros agentes.

Já a noção forte de agente envolve os seguintes aspectos: , veracidade, benevolência
\begin{itemize}
	\item Mobilidade: O Agente deve pode mover-se no ambiente, por exemplo, em uma rede.
	\item Veracidade: Agente não comunica informações falsas.
	\item Benevolência: Agente ajudará os outros.
	\item Racionalidade: O agente não irá agir de forma a impedir a realização de seus objetivos.
	\item Cooperação: O agente coopera com o usuário.
\end{itemize}

\subsection{Arquitetura de agentes}

A arquitetura de agentes varia de acordo com a complexidade da sua autonomia, ou seja, com a capacidade de reagir aos estímulos do ambiente. Conforme verificado no livro de ~\cite{novig95}, os tipos de arquitetura são: orientadas à tabela, reflexiva simples, reflexiva baseado em modelo, baseada em objetivo, baseada em utilidade.

A primeira arquitetura a ser explorada é o agente orientado à tabelas. Todas as ações dos agentes dessa arquitetura são conhecidas e estão gravadas em uma tabela. Assim, quando o agente receber o estímulo ele já terá a ação a ser tomada previamente gravada em sua memória. Logo para construir esse tipo de agente, fica claro que além de saber todas percepções possívels, será necessário definir ações apropriadas para todas. Isso levará a tabelas muito complexas e o tamanho pode facilmente passar a ordem de milhões dependendo do número de entradas.

A arquitetura reflexiva simples é um dos tipos mais simples de agente. Nele, o agente seleciona a ação com base unicamente na percepção atual, desconsiderando assim uma grande tabela de decisões. A decisão é tomada com base de regras condição-ação: Se uma condição ocorrer, uma ação será tomada. Por exemplo, vamos supor um agente médico que determina o diagnóstico de uma doença no paciente caso exista alguma anomalia no organismo (Por exemplo, paciente com febre). Uma condição-ação poderia ser:

if anomalia-organismo then diagnóstico-médico

Esse tipo de agente é bastante simples, o que é uma vantagem comparado à arquitetura de tabela. Porém, essa abordagem requer um ambiente totalmente observável, visto que esse tipo de agente possui uma inteligência bastante limitada. No exemplo do agente médico existem diversas maneiras de se detectar uma anomalia no organismo do paciente, seria necessário conhecer todas as formas para usarmos uma abordagem reativa simples.

A arquitetura reflexiva baseada em modelos funciona de maneira similar a anterior. Nessa abordagem, é levado em conta a parte do ambiente que não é visível neste momento. E para saber o ``momento atual'' de um agente, é necessário guardar a informação de estado consigo. Para atualizar o estado do agente, é necessário conhecer como o mundo desenvolve-se independente do agente (no caso do exemplo, como o organismo funciona) e é necessário saber as ações dos agentes no ambiente. Esses dois conhecimentos do ambiente são chamados de \textbf{modelo do mundo}. O agente que usa esse tipo de abordagem é chamado de agente baseado em modelo.

Na arquitetura reflexiva baseada em objetivo, as ações do agente são tomadas apenas se o aproximam de alcançar um objetivo. Para isso, será necessário algo além do estado atual do ambiente: Será necessário informações do objetivo a ser atingido. Assim o agente pode combinar as informações do estado e o objetivo para determinar se deve ou não agir sobre o ambiente. Essa arquitetura porém é obviamente mais complexa e de certa forma ineficiente. Porém ela permite uma maior flexibilização das ações em determinados ambientes, visto que suas decisões são representadas de forma explícita e podem ser modificadas. É interessante notar que esse tipo de arquitetura não trata ações com objetivos conflitantes.

E por fim, a arquitetura reflexiva baseada em utilidade não utiliza apenas objetivos para realizar a próxima decisão, mas dá ao agente a capacidade de fazer comparações sobre o estado do ambiente e as ações a serem tomadas: Quais delas são mais baratas, confiáveis, baratas, rápidas do que as outras. A capacidade de avaliação do agente chama-se função de utilidade, que mapeia uma sequência de estados em um número real que determina o grau de utilidade. Esse mecanismo possibilita a decisão racional de escolha entre vários objetivos conflitantes. Por exemplo, escolher entre um objetivo mais barato ao invés de escolher entre o mais rápido.

\subsection{Sistemas Multiagentes}

Sistemas multiagentes são sistemas compostos por vários agentes capazes de se comunicar, possuindo uma linguagem de alto nível para isso. O agente deve ser conhecimento para realizar uma determinada tarefa e pode ou não cooperar com outros agentes para realizá-la.

Fica claro nessa definição que sistemas multiagentes

De acordo com~\cite{sarmento11}, podemos encontrar as seguintes características principais de ambientes em SMAs:
\begin{itemize}
	\item Ambientes SMAs fornecem protocolos específicos para comunicação e interação. Cada ambiente tem as suas particularidades: Alguns são em uma única máquina, outros são compartilhados com o mundo real e outros são distribuídos. Cabe a cada ambiente definir um protocolo onde todos agentes devem obedecer para comunicar-se.
	\item SMAs são tipicamente abertos.
	\item SMAs contém agentes que são autônomos e individualistas.
\end{itemize}




























  \chapter{Metodologia de desenvolvimento}

Para a realização deste trabalho, foi planejada uma metodologia de desenvolvimento na qual objetivou-se a construção da arquitetura do SMA por meio de uma metologia de desenvolvimento de SMAs. Inicialmente, foi necessário um levantamento de bibliografias relacionadas à Informática na Educação para o entendimento do problema do trabalho, além de levantamentos de uso de SMA em contextos pegagógicos.

Em seguida foi aplicada, foi realizado estudos aprofundados na metodologia MASE para a implementação deste trabalho. Após o levantamento inicial, seu uso é divido em duas fases, conforme explicado na seção~\ref{section:mase}, sendo a primeira delas essêncial para o levantamento dos requisitos necessários para cumprir os objetivos. A segunda fase foi importante para a distribuição das regras entre os agentes que existem no sistema. As primeiras subseções deverão detalhar a modelagem.

A arquitetura básica do sistema já havia sido previamente decidida~\cite{editalFrank}, sendo necessário detalhar seus requisitos por meio da metodologia MASE. Além disso, os Ambientes Virtuais de Aprendizagem deveriam ser capazes de comunicar-se com a solução proposta, ou seja, a comunicação deve ser feita de forma independente da linguagem de programação. Por fim, foi definido~\cite{editalFrank} o nome da aplicação solução à ser desenvolvida:~\emph{Frank}.

Posteriormente o trabalho seguiu-se com a implementação do SMA na linguagem JAVA, utilizando-se das ferramentas JADE e JBoss Seam. As subseções seguintes deverão detalhar a implementação do Sistema Multiagente e da camada web, bem como a sua integração.

Por fim, a solução foi testada por meio de cenários que simulavam o uso por meio dos atores Aluno e Docente. A demonstração está detalhada no capítulo 4.

\section{A Modelagem}

A modelagem foi desenvolvida utilizando-se a ferramenta~\emph{agentTool}. A ferramenta possui meios para diagramar todas as fases e passos do MASE, auxiliando o analista em todos os diagramas necessários, além de gerar código automático para alguns frameworks de SMA.

Conforme dito a respeito do MASE, a metodologia é dividida em duas fases: Análise e Design. A primeira fase, responsável pelo levantamento de requisitos e entendimento das regras e tarefas, é apresentada na subseção~\ref{subsection:analise}. A segunda fase está apresentada em~\ref{subsection:design}.

\subsection{Análise}\label{subsection:analise}

A metodologia inicia-se com o passo de captura das metas. Para tanto, foi necessário primeiramente um levantamento inicial dos requisitos do SMA. Os requisitos foram levantados e compreendidos por meio de~\cite{editalFrank}, onde é possível listar:

\begin{enumerate}
	\item O sistema deve manter um modelo do estudante, onde será determinado o seu estilo de aprendizagem e será notíficado ao docente.
	\item O sistema deve assistir (auxiliar) o aluno por meio de um grupo de trabalho.
	\item O sistema deve fazer interface com Ambiente Virtual de Aprendizagem, a fim de estabelecer comportamentos do estudante.
	\item O sistema deve criar uma modelagem cognitiva do aluno, onde são mantidas informações sobre o desempenho, de acordo sua interação em Ambiente Virtual de Aprendizagem, e informações a respeito do seu estilo de aprendizagem.
	\item O sistema deve criar uma modelagem metacognitiva do aluno, onde são armazenadas informações com o intuito de melhorar processos de aprendizagem de domínios específicos.
	\item O sistema deve criar uma modelagem afetiva do estudante, especificamente a respeito da modelagem da personalidade e emoções do estudante.
	\item O sistema deve fazer interface com Ambiente Virtual de Aprendizagem.
	\item O sistema deve refutar ou confirmar o estilo de aprendizagem do aluno a partir do desempenho relacionado à interação com o sistema e/ou com Ambiente Virtual de Aprendizagem.
	\item O sistema SMA deve atualizar o modelo do estudante com base em inferências a partir dos registros de trabalho do estudante.
	\item O sistema deve construir o modelo do estudante a partir de uma modelagem explícita, ou seja, a partir do feedback explícito do estudante (questionário).
	\item O sistema deve construir o modelo do estudante a partir de uma modelagem implícita, ou seja, a partir do desempenho obtido nas ferramentas de aprendizado.
\end{enumerate}

A partir dos requisitos de~\cite{editalFrank}, foi possível estabelecer metas que o sistema deveria atingir para satisfazê-los:

\begin{itemize}
	\item Manter um modelo do estudante (Meta geral do sistema)
	\item Notificar ao docente
	\item Auxiliar o aluno por meio de um grupo de trabalho
	\item Criar modelagem cognitiva
	\item Criar modelagem metacognitiva
	\item Criar modelagem modelagem afetiva
	\item Criar modelagem da personalidade
	\item Criar modelagem das emoções do estudante
	\item Interface com ferramentas de aprendizagem dedicada
	\item Confirmar estilo de aprendizagem do aluno
	\item Refutar estilo de aprendizagem do aluno
	\item Construir modelo de desempenho do aluno
	\item Construir modelagem explícita
	\item Construir modelagem implícita
	\item Construir modelo de estilo de aprendizagem do aluno
\end{itemize}

A partir do levantamento, foi possível observar que a meta "Manter um modelo do estudante" abrange o escopo geral de toda a aplicação, sendo possível estabelecer como meta do sistema. Em seguida, foi necessário hierarquizar as metas de forma a encontrar quais metas poderiam ser estabelecidas com o cumprimento de outras. A imagem~\ref{fig:metas-frank} representa a hierarquia de metas que foi estabelecida. Os retângulos em cinza representam metas particionadas.

\begin{sidewaystable}
	\centering
	\includegraphics[scale=0.8]{images/metas-frank.png}
	\caption{Hierarquia de Metas do SMA Frank.}
	\label{fig:metas-frank}
\end{sidewaystable}

O segundo passo da metodologia consiste no desenvolvimento dos casos de uso. Foram levantados 5 principais casos de uso, alguns com fluxos alternativos que representam situações opcionais no caso de uso. Este trabalho utiliza-se da notação completa de desenvolvimento de casos de uso.

O primeiro caso de uso diz respeito à modelagem cognitiva do aluno. Existem dois cenários possíveis: Modelagem implícita (principal cenário de sucesso) e modelagem explícita (cenário alternativo). Basicamente o SMA deverá processar o questionário de estilos de aprendizagem, respondido pelo aluno, para inferir explicitamente o seu modelo cognitivo e deverá analisar as respostas enviadas por ele para inferir explicitamente o seu modelo cognitivo.

\begin{usecase}

\addtitle{Caso de Uso 1}{Determinar Modelagem Cognitiva do Aluno}

\addfield{Escopo:}{Sistema e Ambiente(s) Virtuais(ns) de Aprendizagem(s)}
\addfield{Nível:}{Objetivo do sistema}
\addfield{Ator Principal:}{Sistema}

\additemizedfield{Interessados e Interesses:}{
	\item Sistema: Manter um modelo do estilo de aprendizagem do estudante a partir dos dados gerados do formulário, bem como com algum Ambiente Virtual de Aprendizagem.
	\item Aluno: Preencher o questionário de estilos de aprendizagem e interagir com um AVA.
}

\addfield{Pré-condições:}{Não se aplica}
\addfield{Pós-condições:}{}

\addscenario{Cenário de Sucesso Principal:}{
	\item O Aluno faz o primeiro login no sistema.
	\item O Sistema reconhece o perfil do aluno e cria um Grupo de Trabalho (GT) de agentes.
	\item O Aluno deverá preencher o questionário de estilos de aprendizagem.
	\item O Sistema avalia as respostas preenchidas pelo aluno e, conforme a literatura de psicologia, determinará o estilo de aprendizagem por meio da avaliação explícita.
	\item O Sistema construirá o primeiro modelo do aluno, com base no seu estilo de aprendizagem inicial.
	\item O Sistema salva os dados, a fim de notificar o docente dos dados obtidos.
	\item O Sistema notifica o Aluno sobre o seu estilo de aprendizagem.
	\item O Sistema exibe a lista de AVA disponíveis para o Aluno.
}

\addscenario{Extensões:}{
	\item[1.a] Falha no login:
		\begin{enumerate}
		\item[1.] O Sistema mostra mensagem de falha.
		\item[2.] O Docente retorna ao passo 1.
		\end{enumerate}
	\item[2.a] Inferência implícita de estilo de aprendizagem:
		\begin{enumerate}
		\item[1.] O Aluno faz login no sistema
		\item[2.] O Sistema reconhece o perfil do aluno e cria um grupo de trabalho de agentes.
		\item[3.] O Sistema exibe a lista de AVA disponíveis para o Aluno.
		\item[4.] O Aluno seleciona o AVA
		\item[5.] O Sistema é notificado do AVA escolhido
		\item[6.] O AVA registra as ações do Aluno e envia para o Sistema.
		\item[7.] O AVA interage com o Sistema por meio do caso de uso.
		\item[8.] O Sistema utiliza os resultados de atividades feitas pelo aluno para inferir o seu estilo de aprendizagem, levando em consideração a literatura da psicologia relativa à atividade desenvolvida.
		\item[9.] O Sistema compara então os dados obtidos com os dados previamente armazenados, a fim de obter um grau de certeza do estilo de aprendizagem que foi construído.
		\item[10.] O Sistema salva os dados, a fim de notificar o Docente dos dados obtidos.
		\end{enumerate}
	\item[3.a] Inferência implícita de estilo de aprendizagem:
		\begin{enumerate}
		\item[1.] O Sistema analisa os dados obtidos do ambiente para determinar o desempenho do Aluno na atividade, levando em consideração a literatura pedagógica relativa.
		\item[2.] O Sistema salva os dados no banco de dados, a fim de notificar o Docente dos dados obtidos.
		\end{enumerate}
}

\addscenario{Requisitos Especiais:}{
	\item Não se aplica
}

\addscenario{Lista de Variantes Tecnologias de Dados:}{
	\item Não se aplica
}

\addfield{Frequência de Ocorrência:}{Sempre}

\end{usecase}


O segundo caso de uso descreve o cenário de notificação do docente. Nele, o docente é autenticado no sistema e o sistema exibe uma lista de alunos disponíveis nas mais diversas turmas. O docente seleciona um aluno e então o sistema exibe os dados relativos ao modelo do aluno. O trecho possui a descrição do caso de uso.

\begin{usecase}

\addtitle{Caso de Uso 2}{Notificar Docente} 

\addfield{Escopo:}{Todo o Sistema}
\addfield{Nível:}{Objetivo do usuário}
\addfield{Ator Principal:}{Docente}

\additemizedfield{Interessados e Interesses:}{
	\item Docente: Deseja ser notificado sobre o modelo do estudante.
}

\addfield{Pré-condições:}{Não se aplica}
\addfield{Pós-condições:}{O Docente deve visualizar o modelo cognitivo do Aluno}

\addscenario{Cenário de Sucesso Principal:}{
	\item O Docente faz o login no sistema.
	\item O Sistema recupera a lista de alunos disponíveis.
	\item O Sistema exibe a lista de alunos.
	\item O Docente seleciona o aluno desejado.
	\item O Sistema requisita o modelo do estudante ao SMA.
	\item O SMA exibe as informações do modelo
	\item O Sistema carrega as seguintes informações relativas à modelagem do aluno: Modelagem cognitiva, Modelagem metacognitiva e Modelagem afetiva.
}

\addscenario{Extensões:}{
	\item[1.a] Falha no login:
		\begin{enumerate}
		\item[1.] O Sistema mostra mensagem de falha.
		\item[2.] O Docente retorna ao passo 1.
		\end{enumerate}
}

\addscenario{Requisitos Especiais:}{
	\item Não se aplica
}

\addscenario{Lista de Variantes Tecnologias de Dados:}{
	\item Não se aplica
}

\addfield{Frequência de Ocorrência:}{Sempre}

\end{usecase}


Os terceiro e quarto casos de uso dizem respeito à inferência do modelo afetivo e metacognitivo do aluno, respectivamente.

\input{capitulos/casos-uso/uc3}

\begin{usecase}

\addtitle{Caso de Uso 4}{Determinar Modelagem Metacognitiva do Aluno} 

\addfield{Escopo:}{Sistema e Ambiente(s) Virtuais(ns) de Aprendizagem(s)}
\addfield{Nível:}{Objetivo do sistema}
\addfield{Ator Principal:}{Sistema}

\additemizedfield{Interessados e Interesses:}{
	\item Sistema: Criar e manter um modelo metacognitivo do estudante a partir dos dados gerados da sua interação com um Ambiente Virtual de Aprendizagem (AVA).
	\item Aluno: Visualizar o modelo afetivo determinado pelo Sistema.
}

\addfield{Pré-condições:}{}
\addfield{Pós-condições:}{}

\addscenario{Cenário de Sucesso Principal:}{
	\item O Aluno faz login no Sistema.
	\item O Sistema exibe a lista de AVAs disponíveis para o Aluno.
	\item O Aluno seleciona o AVA.
	\item O Sistema é notificado do AVA escolhido
	\item O AVA registra as ações do Aluno e envia para o Sistema.
	\item O ambiente interage com o Sistema por meio do caso de uso UC5
	\item O Sistema utiliza os resultados de atividades feitas pelo aluno para criar um modelo metacognitivo do estudante, levando em consideração a literatura da psicologia relativa à atividade desenvolvida.
	\item O Sistema compara então os dados obtidos com os dados previamente armazenados no banco de dados, caso existam, a fim de atualizar o modelo previamente criado.
	\item O Sistema salva os dados no banco de dados, a fim de notificar o Docente dos dados obtidos.
}

\addscenario{Extensões:}{
	\item[1.a] Falha no login:
		\begin{enumerate}
		\item[1.] O Sistema mostra mensagem de falha.
		\item[2.] O Docente retorna ao passo 1.
		\end{enumerate}
}

\addscenario{Requisitos Especiais:}{
	\item Não se aplica
}

\addscenario{Lista de Variantes Tecnologias de Dados:}{
	\item Não se aplica
}

\addfield{Frequência de Ocorrência:}{Sempre}

\end{usecase}


Por fim, o último caso de uso foi levantado para prever a interação do AVA com o SMA Frank. Devido a possibilidade dos Ambientes Virtuais de Aprendizagem serem desenvolvidos em qualquer linguagem, é necessário utilizar-se de uma forma de comunicação comum entre aplicações.

Logo o SMA Frank irá utilizar-se de WebServices para a comunicação externa, garantindo que diversas aplicações poderão interagir com o SMA. Para novos AVAs, tudo o que precisará ser feito é a implementação da assinatura do serviço no WebService. Dessa forma a solução garante uma intervenção mínima no código do AVA, exigindo menos tempo na codificação da comunicação e garantindo o foco na inferência à ser feita pelo SMA. Segue a descrição do caso de uso.

\begin{usecase}

\addtitle{Caso de Uso 5}{Determinar Modelagem Metacognitiva do Aluno} 

\addfield{Escopo:}{Sistema e Ambiente(s) Virtuais(ns) de Aprendizagem(s)}
\addfield{Nível:}{Objetivo do sistema}
\addfield{Ator Principal:}{Aluno}

\additemizedfield{Interessados e Interesses:}{
	\item Sistema: Obter os dados do Aluno após a interação com o Ambiente Virtual de Aprendizagem.
	\item Ambiente Virtual de Aprendizagem: Comunicar o Sistema sobre os resultados relativos ao desenvolvimento de uma atividade do Aluno.
}

\addfield{Pré-condições:}{O Aluno deve estar autenticado no Sistema.}
\addfield{Pós-condições:}{}

\addscenario{Cenário de Sucesso Principal:}{
	\item O Ambiente Virtual de Aprendizagem registra as ações do Aluno.
	\item O AVA envia os resultados do aluno via web-service para o Sistema SMA, a partir de uma URL disponível no ambiente. Os dados serão enviados, bem como um token de autenticação que identifica o Aluno e o seu grupo de trabalho no sistema.
	\item O Sistema SMA recebe os dados enviados pelo Ambiente.
	\item O Sistema valida os dados.
	\item Os dados são enviados para o devido grupo de trabalho do Aluno.
}

\addscenario{Extensões:}{
	\item Não se aplica.
}

\addscenario{Requisitos Especiais:}{
	\item Não se aplica.
}

\addscenario{Lista de Variantes Tecnologias de Dados:}{
	\item Não se aplica.
}

\addfield{Frequência de Ocorrência:}{Sempre}

\end{usecase}


Após o desenvolvimento dos casos de uso, foi necessário refinar os diagramas de sequência. Todos os diagramas foram desenvolvidos na ferramenta~\emph{agentTool}, visto que ele acompanha todas as fases do MASE. Para o primeiro caso de uso, foram desenvolvidos dois diagramas de sequência distintos: Um para o fluxo principal e outro para o fluxo alternativo.

A imagem~\ref{fig:dss-uc1-fluxo-principal} refere-se ao fluxo principal do caso de uso 1. Neste diagrama de sequência, existem 6 regras. O fluxo do Sistema inicia-se com a regra~\emph{Manager}. Ele gera um evento de localização do aluno. Em seguida, gera o evento enviar para a regra~\emph{StudentWorkGroup}, com o parâmetro~\emph{questionário}. A regra~\emph{StudentWorkGroup} gera o evento de enviar para as regras~\emph{CognitiveAction},~\emph{AffectiveAction} e~\emph{MetacognitiveAction}. Eles retornam respectivamente os modelos~\emph{Cognitivo},~\emph{Afetivo} e~\emph{Metacognitivo}. A regra~\emph{cognitiveAction} ainda gera mais um evento de envio para a regra~\emph{LearningMethodAnalyzer}, que retorna o estilo de aprendizagem.

\begin{figure}
	\centering
	\includegraphics[scale=0.48]{images/dss-uc1-fluxo-principal.png}
	\caption{Diagrama de sequência do fluxo principal, caso de uso 1.}
	\label{fig:dss-uc1-fluxo-principal}
\end{figure}

A imagem~\ref{fig:dss-uc1-fluxo-alternativo} representa o fluxo de exceção do primeiro caso de uso. A regra~\emph{WebServiceInterface} recebe os dados do AVA e envia para a regra Manager por meio do evento~\emph{enviar}. Após esse evento, a regra~\emph{StudentWorkgroup} recebe o evento e reenvia para~\emph{CognitiveAction}. Em seguida ele envia para as regras~\emph{LearningMethodAnalyzer} e~\emph{PerformanceAnalyzer} que vão inferir o estilo de aprendizagem e a performance. Por fim, com estes dados, o modelo cognitivo é retornado para a regra~\emph{StudentWorkgroup}.

\begin{figure}
	\centering
	\includegraphics[scale=0.48]{images/dss-uc1-fluxo-alternativo}
	\caption{Diagrama de sequência do fluxo de exceção, caso de uso 1.}
	\label{fig:dss-uc1-fluxo-alternativo}
\end{figure}

A imagem~\ref{fig:dss-uc3-fluxo-principal} representa o fluxo principal do caso de uso 3, inferência afetiva. O processo de comunicação das regras~\emph{WebServiceInterface} e~\emph{StudentWorkgroup} funciona de forma semelhante ao diagrama anterior. A regra~\emph{AffectiveAction} gera um evento de inferência de modelagem afetiva.

\begin{figure}
	\centering
	\includegraphics[scale=0.48]{images/dss-uc3-fluxo-principal.png}
	\caption{Diagrama de sequência do fluxo principal, caso de uso 3.}
	\label{fig:dss-uc3-fluxo-principal}
\end{figure}

O diagrama de sequência 4~\ref{fig:dss-uc4-fluxo-principal} funciona similar ao caso de uso anterior, com a diferença de que a regra~\emph{MetacognitiveAction} realiza a inferência da modelagem metacognitiva.

\begin{figure}
	\centering
	\includegraphics[scale=0.48]{images/dss-uc4-fluxo-principal.png}
	\caption{Diagrama de sequência do fluxo principal, caso de uso 4.}
	\label{fig:dss-uc4-fluxo-principal}
\end{figure}

Por fim o último diagrama de sequência~\ref{fig:dss-uc5-fluxo-principal} representa a counicação da regra~\emph{WebServiceInterface} com a regra~\emph{Manager}. A primeira realiza a validação de dados e em seguida o envio de dados. Após receber os dados, a regra~\emph{Manager} localiza o aluno e continua o fluxo de execução.

\begin{figure}
	\centering
	\includegraphics[scale=0.48]{images/dss-uc5-fluxo-principal.png}
	\caption{Diagrama de sequência do fluxo principal, caso de uso 5.}
	\label{fig:dss-uc5-fluxo-principal}
\end{figure}

Por fim, após o levantamento de todas as regras foi necessário criar tarefas para a criação do~\emph{MASE Role Model}.


\begin{table}
	\caption{Estruturação das Tarefas por Regra}
	\begin{tabular}{|p{5cm} | p{9cm}|}
		\hline
		\textbf{Seção do Caso de Uso}	& \textbf{Significado} \\
		\hline
		Nome do Caso de Uso 	& Nome do caso de uso, iniciando-se com um verbo  \\
		\hline
	\end{tabular}
	\label{tabela:topicos_uc}
\end{table}

\begin{figure}
	\centering
	\includegraphics[scale=0.48]{images/mase-role-model.png}
	\caption{Diagrama ~\emph{MASE Role Model} gerado para o SMA Frank.}
	\label{fig:dss-uc5-fluxo-principal}
\end{figure}







\subsection{Design}\label{subsection:design}

  \chapter{Testes e Simulações}

\section{Demonstração da Interface com Aluno}

\section{Demonstração da Interface com Docente}

  \chapter{Conclusões e Trabalhos Futuros}
Este capítulo analisa o trabalho realizado discutindo o alcance dos objetivos por meio da arquitetura e protótipo desenvolvidos. Em seguida, são indicados uma série de possíveis trabalhos futuros em cima da plataforma Frank.

Este trabalhou visou o auxílio de docentes na observação de seus alunos, bem como melhoria na sua didádica de ensino ao observar a melhor forma que seus alunos podem absorver os conhecimentos adquiridos em sala de aula. Em face disso, a arquitetura proposta objetivou traçar os modelos cognitivo,  afetivo e metacognitivo do aluno, criando insumos para o professor desenvolver melhor suas didáticas de ensino.

Ao fim deste trabalho, foi possível definir e implementar a arquitetura geral do SMA Frank:
\begin{itemize}
 	\item Dois agentes de interfaces com o ambiente externo;
	\item Um agente para controle do ambiente do SMA;
	\item Um grupo de trabalho para cada aluno, composto por 4 agentes: O grupo de trabalho e os agentes cognitvo, metacognitivo e afetivo.
\end{itemize}

Além disso, a implementação do protótipo da plataforma web é capaz de interagir tanto com os alunos, quanto com os seus professores.

Para tanto, com os objetivos alcançados, concluí-se que este trabalho possui uma potêncial importância para a sociedade como um todo, pois auxilia o complexo e longo processo de aprendizagem de alunos e possui capacidade para a melhoria significativa do trabalho do docente.

A arquitetura do SMA Frank está proposta, porém é necessária a continuação de diversos aspectos na arquitetura. Primeiramente, como trabalho futuro faz-se necessário obter a modelagem metacognitiva e afetiva dos alunos, além das suas respectivas formas de inferências implícitas.

Como trabalho futuro, é necessário uma integração com um ambiente de aprendizagem dedicado e realizar testes reais com alunos e docentes para a validação das inferências realizadas.

Por fim, é desejável que a plataforma web possua um controle maior do ambiente SMA. Ou seja, uma administração web que controle todo o SMA, sendo possível a criação e remoção remota de agentes, o balanceamento dos agentes entre as diversas máquinas disponibilizadas para aplicação, a visualização do estado atual de cada grupo de trabalho de cada aluno.


  \postextual
  \bibliographystyle{plain}
  \bibliography{bibliografia}

\end{document}
