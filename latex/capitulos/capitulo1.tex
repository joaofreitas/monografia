\chapter{Introdução}

Com o crescente desenvolvimento da computação que penetra cada vez mais em diversas áreas de conhecimento, a demanda por processamento tende a crescer rapidamente, tornando o desenvolvimento de aplicações cada vez mais complexo, exigindo cada vez mais desempenho e consequentemente poder de processamento. Para a execução dessas aplicações em tempo hábil, são necessários investimentos cada vez mais altos em \emph{hardwares} melhores, existindo porém um fator limitante (custo ou tecnologia).

Além disso, atualmente a grande quantidade de informação disponível exige análises cada vez mais precisas e detalhadas da informação processada tendo em vista a ajuda de tomada de decisões. Dessa forma, aplicações que antes eram centralizadas em uma única máquina transformaram-se em aplicações distribuídas em várias máquinas que são concorrentes e assíncronas.

A partir dessa motivação, aplicações são projetadas para rodar de forma descentralizada, com componentes e serviços rodando em diversos lugares distintos e comunicando-se uma com as outras através de mensagens. Cada módulo pode ter um objetivo específico, como capturar e processar eventos no ambiente computacional, processar informações, persistir eventos no banco de dados, enfim, uma vasta gama de operações que são assíncronas e independentes. A arquitetura dessas aplicações é projetada objetivando o alto paralelismo, flexibilidade, interoperabilidade, dentre outros aspectos.

Diversos~\emph{frameworks} tentam lidar com o problema da computação distribuída, porém alguns usam arquiteturas que são baseadas em processamento ordenado: Os dados são processados ordenadamente, não usufruindo de todo o processamento que poderia ocorrer se fosse realmente paralelo. Outros ~\emph{frameworks} são embasados em tecnologias que não são recomendáveis em um ambiente distribuído: O uso de recurso um compartilhado e blocante, que pode prejudicar o processamento em larga escola.

A computação distribuída toma formas ainda mais interessantes quando aplicada a contextos sensíveis a sociedade em geral. Áreas de atuação como Bolsa de Valores, Análise de Redes Sociais, Análise de Mídia Social, Informática na Educação, dentre outras. Em especial a esta última área, a possibilidade de auxílio no aprendizado do aluno eleva a importância deste setor na computação.

Na perspectiva da Informática na Educação (IE), a abordagem chamada Sistemas Tutores Inteligentes permite a representação de conhecimento de forma muito interessante. É possível a construção um modelo onde se representa o estudante (o objeto a quem se deve ensinar), o domínio do conteúdo (o conteúdo a ser ensinado) e o modelo pedagógico (a forma a ser ensinada). Esta abordagem permite o uso de vários conceitos da Inteligência Artificial para determinação de estilos de aprendizagem, dentre outras possibilidades. Dessa forma, aplicações tendem a ser bastante complexas vistas ao alto grau de processamento que este ambiente pode assumir.

\section{Problema}
Os ambientes educacionais de aprendizagem não possuem inferência de modelo do estudante, pois sua implementação seguindo modelo cliente-servidor não é apropriada para tal finalidade. Além disso, a alta carga da aplicação devido ao fato de muitos acessos dos alunos pode prejudicar uma aplicaçao centralizada, necessitando de um sistema que processasse em múltiplos locais.

\section{Objetivos do Projeto}
Tendo em vista o cenário atual apresentado, o objetivo geral deste trabalho é propor a arquitetura distribuída de um ambiente computacional, bem como sua construção, baseada em Sistemas Multiagentes, capaz de construir e manter um modelo do estudante, a partir do qual os estilos de aprendizagens desse estudante poderão ser identificados e informados ao docente por meio de uma interface web.

Os objetivos específicos serão devidamente justificados na próxima subseção, bem como idealizada a forma de atingi-los.

Usando tecnologias existentes e consolidadas, a arquitetura proposta irá usar o framework~\emph{JADE}, que é completamente desenvolvido na linguagem~\emph{JAVA} e simplifica a implementação de Sistemas Multiagentes (SMA) que cumprem as especificações FIPA. A arquitetura proposta também englobará uma interface web que utiliza a plataforma~\emph{open source JBoss Seam}, desenvolvida para auxiliar a construção de aplicações dinâmicas para a internet de forma simples e ágil.

\section{Objetivos Específicos do Projeto}
Este trabalho foi norteado pelo documento do projeto Frank~\cite{editalFrank}, sendo construído como uma proposta ao problema nele proposto. A partir deste documento, é possível identificar os objetivos específicos derivados da construção do ambiente computacional para manutenção do modelo do aluno. Mais especificamente, os objetivos específicos deste trabalho são:
 \begin{itemize}
 	\item Objetivo específico 1: Obter uma modelagem da arquitetura geral do SMA Frank utilizando-se da metodologia~\emph{Multiagent System Engineering} (MASE), proposta como uma solução de Engenharia de Software para o desenvolvimento de SMA;
	\item Objetivo específico 2: Obter uma implementação da arquitetura geral do SMA Frank;
	\item Objetivo específico 3: Obter uma implementação da arquitetura dos agentes assistentes de cognição, metacognição e afetivo;
	\item Objetivo específico 4: propor uma interface do agente assistente de cognição com o estudante;
	\item Objetivo específico 5: propor uma interface do agente assistente de cognição com o docente;
\end{itemize}

A seguinte estrutura desse trabalho consiste na divisão de capítulos visando facilitar a leitura e organizar os conceitos que perfazem o desenvolvimento deste trabalho:
\begin{itemize}
	\item O presente capítulo contém a introdução, onde o trabalho é justificado e os objetivos são esclarecidos.
	\item Capítulo 2 contém todos os fundamentos teóricos necessários para o desenvolvimento desse trabalho.
	\item Capítulo 3 contém a proposta de solução composta pela metodologia, modelagem da arquitetura e implementação.
	\item Capítulo 4 contém a experimentação e análise de resultados.
	\item Por fim, o capítulo 5 relata a conclusão e trabalhos futuros.
\end{itemize}
