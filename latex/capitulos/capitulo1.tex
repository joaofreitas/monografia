\chapter{Introdução}

Com o crescente desenvolvimento da computação que penetra cada vez mais em diversas áreas de conhecimento, a demanda por processamento tende a crescer rapidamente, tornando o desenvolvimento de aplicações cada vez mais complexo, exigindo cada vez mais desempenho e consequentemente poder de processamento. Para a execução dessas aplicações em tempo hábil, são necessários investimentos cada vez mais altos em \emph{hardwares} melhores, existindo porém um fator limitante (custo ou tecnologia).

Além disso, atualmente a grande quantidade de informação disponível exige análises cada vez mais precisas e detalhadas da informação processada tendo em vista a ajuda de tomada de decisões. Dessa forma, aplicações que antes eram centralizadas em uma única máquina transformaram-se em aplicações distribuídas em várias máquinas que são concorrentes e assíncronas.

A partir dessa motivação, aplicações são projetadas para rodar de forma descentralizada, com componentes e serviços rodando em diversos lugares distintos e comunicando-se uma com as outras através de mensagens. Cada módulo pode ter um objetivo específico, como capturar e processar eventos no ambiente computacional, processar informações, persistir eventos no banco de dados, enfim, uma vasta gama de operações que são assíncronas e independentes. A arquitetura dessas aplicações é projetada objetivando o alto paralelismo, flexibilidade, interoperabilidade, dentre outros aspectos.

Diversos~\emph{frameworks} tentam lidar com o problema da computação distribuída, porém alguns usam arquiteturas que são baseadas em processamento ordenado: Os dados são processados ordenadamente, não usufruindo de todo o processamento que poderia ocorrer se fosse realmente paralelo. Outros ~\emph{frameworks} são embasados em tecnologias que não são recomendáveis em um ambiente distribuído: O uso de recurso um compartilhado e blocante, que pode prejudicar o processamento em larga escola.

A computação distribuída toma formas ainda mais interessantes quando aplicada a contextos sensíveis a sociedade em geral. Áreas de atuação como Bolsa de Valores, Análise de Redes Sociais, Análise de Mídia Social, Informática na Educação, dentre outras. Em especial a esta última área, a possibilidade de auxílio no aprendizado do aluno eleva a importância deste setor na computação.

Na perspectiva da Informática na Educação (IE), a abordagem chamada Sistemas Tutores Inteligentes permite a representação de conhecimento de forma muito interessante. É possível a construção um modelo onde se representa o estudante (o objeto a quem se deve ensinar), o domínio do conteúdo (o conteúdo a ser ensinado) e o modelo pedagógico (a forma a ser ensinada). Esta abordagem permite o uso de vários conceitos da Inteligência Artificial para determinação de estilos de aprendizagem, dentre outras possibilidades. Dessa forma, aplicações tendem a ser bastante complexas vistas ao alto grau de processamento que este ambiente pode assumir.

\section{Problema}
Os ambientes educacionais de aprendizagem não possuem inferência de modelo do estudante, pois sua implementação seguindo modelo cliente-servidor não é apropriada para tal finalidade. Além disso, a alta carga da aplicação devido ao fato de muitos acessos dos alunos pode prejudicar uma aplicaçao centralizada, necessitando de um sistema que processasse em múltiplos locais.


\section{Objetivos do Projeto}
Tendo em vista o cenário atual apresentado, o objetivo geral deste trabalho é construção de um ambiente computacional, baseado em Sistemas Multiagentes, capaz de construir e manter um modelo do estudante, a partir do qual os estilos de aprendizagens desse estudante poderão ser identificados e informados ao docente.

Os objetivos específicos serão devidamente justificados na próxima subseção, bem como idealizada a forma de atingi-los.

Usando tecnologias existentes e consolidadas, a arquitetura proposta irá usar a linguagem de programação~\emph{Akka}, desenvolvida para lidar com processamento paralelo e de alta escalabilidade com abordagem de atores, bem como o~\emph{framework} ~\emph{Akka} para simplificar o desenvolvimento de aplicações escaláveis, concorrentes e tolerantes a falhas, abstraindo esse processamento de forma simples e limpa e combinando essas ferramentas em uma arquitetura de Sistemas Multiagentes.

\section{Uma justificativa para o desenvolvimento do SMA Frank}
Conforme o objetivo do edital~\cite{editalFrank}, deverá ser construído o ambiente computacional para manutenção do modelo do aluno. Conforme será demonstrado as características das ferramentas escolhidas no decorrer deste trabalho, a combinação Scala (Abordagem de Atores), Akka em uma arquitetura de Sistemas Multiagentes pode tornar-se numa poderosa plataforma para abrigar este componente computacional.

Mais especificamente, os objetivos específicos do trabalho poderão ser atingidos da seguinte forma:
 \begin{itemize}
	\item Objetivo específico 1: A arquitetura geral deverá ser implementada no framework Akka, que fará a distribuição dos agentes em vários nós e processadores. A linguagem Scala será usada utilizada, pois será utilizada uma abordagem de atores que é nativa da linguagem.
	\item Objetivo específico 2: Os diversos atores do Scala podem manter o estado do aprendizado do aluno e comunicar-se com o ambiente, que terá arquitetura de SMA, após a sua mudança de estado. Outros atores podem fazer a análise do desempenho do aluno, inferências de estilo de aprendizagem, persistência de dados em banco, interface web com usuário, dentre outras funcionalidades. Em resumo, o desenvolvimento do agente/ator de cognição na arquitetura em Scala + Akka torna-se viável.
	\item Objetivo específico 3 e 4: Será implementado uma interface web com a linguagem Scala e o framework Play~\cite{playFramework} para desenvolvimento web, devido à seu código ser desenvolvido em Scala.
	\item Objetivo específico 5: A interface com um ambiente interativo de aprendizagem deverá ser analisada e modelada durante o desenvolvimento do trabalho de graduação. Considerando a ferramenta~\cite{quizWiki}, por exemplo, essa interface poderá ser implementada por meio Banco de Dados, ou definindo um~\emph{schema} para as duas ferramentas se comunicarem ou mesmo uma integração entre as duas linguagens.
\end{itemize}


A seguinte estrutura desse trabalho consiste na divisão de capítulos visando facilitar a leitura e organizar os conceitos que perfazem o desenvolvimento deste trabalho:
\begin{itemize}
	\item Capítulo 1 contém a introdução.
	\item Capítulo 2 contém todos os fundamentos teóricos necessários para o desenvolvimento desse trabalho.
	\item Capítulo 3 contém a proposta de solução composta pela metodologia, modelagem da arquitetura e implementação.
	\item Capítulo 4 contém a experimentação e análise de resultados.
	\item Por fim, o capítulo 5 relata a conclusão e trabalhos futuros.
\end{itemize}
