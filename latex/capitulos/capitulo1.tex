\chapter{Introdução}

Cada pessoa possui uma forma preferencial de absorção do conhecimento. Seja por imagens, textos, teoria ou prática, durante uma situação de aprendizagem todos tendem a receber e processar melhor as informações que são recebidas de certa maneira, em detrimento a outras. Essa forma de recepção do conhecimento é nomeada estilo de aprendizagem~\cite{felder1988learning}.
 
O acesso à recursos tecnológicos, outrora caros e difíceis, tornaram-se presentes no cotidiano de muitas pessoas devido à facilidade de aquisição. Dessa forma, a viabilidade do aprendizado por meio do computador aumentou e começa a modificar o paradigma do professor detentor do conhecimento e o único responśavel pela sua transmissão.

Em um ambiente escolar a forma didática escolhida por um docente pode afetar o desempenho dos seus alunos, visto que a forma de transmissão do conhecimento escolhida pode prejudicar alguns estudantes. A situação ideal requer a existência de uma personalização do ensino para cada estudante, considerando as características inerentes à sua cognição. Inspirado por esse ideal, algumas ferramentas no âmbito da computação pretendem o ensino personalizado aos estudantes de acordo com o seu perfil de utilização.

Assim, conhecer os fatores relacionados ao processo de aprendizagem exige que as ferramentas computacionais aplicadas ao ensino consigam determinar eficientemente os estilos de aprendizagem. Porém, a complexidade das aplicações aumenta devido às abordagens que serão empregadas, bem como o aumento da exigência de processamento.

Algumas abordagens visam diminuir essa complexidade, permitindo a representação aproximada das características dos alunos em ambientes computacionais, de forma que seja possível representar diversos domínios de modelo para um único estudante. Por exemplo, o modelo do estudante elaborado na forma multidimensional a partir dos universos cognitivo, metacognitivo e afetivo.

Somado à isso, técnicas de computação distribuída são projetadas para a descentralização de recursos, com componentes e serviços executando em lugares distintos e comunicando-se por meio de mensagens. A arquitetura dessas aplicações é projetada objetivando o alto paralelismo, extensibilidade, interoperabilidade, dentre outros aspectos, permitindo a coexistência de diversos modelos de estudantes em uma única aplicação.

Portanto, determinar o estilo de aprendizagem mostra-se uma estratégia fundamental para que a transmissão de informações e vivências entre alunos e professores torne-se mais eficaz e perceptível, promovendo a criação de informações cada vez mais relevantes para o planejamento, acompanhamento e avaliação dos aprendizes.

\section{Contextualização}
Este trabalho agradece o Decanato de Ensino de Graduação (DEG) - UnB, visto que originou-se na proposta elaborada em~\cite{editalFrank} para atender ao edital do projeto divulgado pelo decanato. A proposta em questão contextualiza toda a problemática e estrutura um ambiente para manutenção do modelo do aluno com abordagem de agentes, bem como define as ferramentas a serem utilizadas.

Baseando-se na proposta, este trabalho detalha a solução dos agentes e justifica a arquitetura por meio de uma metodologia específica, bem como realiza um estudo de ferramentas a serem utilizadas na elaboração da solução.

\section{Problema}
A análise do modelo multidimensional do aluno identifica suas características particulares, compreendendo os modelos cognitivo, metacognitivo e afetivo, tornando-se informações fundamentais para o docente na busca pela didática mais recomendável para seus alunos. Diversas arquiteturas porém lidam com o problema de distintas formas, algumas por exemplo lidam com o problema de forma simples, outras não permitem a representação do estilo de aprendizagem do aluno em tempo real.

Considerando este cenário, algumas perguntas formulam o problema: Qual arquitetura possibilitaria a representação individualizada do aluno em tempo real para a interação entre modelos de estudantes e estilos de aprendizagem? Esta arquitetura ocasionará uma maior complexidade na inferência do estilo de aprendizagem?

\section{Objetivos}
Tendo em vista o problema apresentado, o presente trabalho tem como objetivo definir uma arquitetura distribuída na~\emph{web} para auxiliar o processo de ensino-aprendizagem por meio da abordagem de Sistema multiagente (SMA), criando insumos para auxiliar o docente na adoção da melhor estratégia didática de ensino.

Esta abordagem permitirá a construção e manutenção de um modelo multidimensional do estudante, a partir do qual os estilos de aprendizagem desse estudante poderão ser identificados. A abordagem SMA permitirá a representação do conhecimento do aluno por meio de agentes. Além disso, é possível a decomposição do problema na modelagem multidimensional em vários subproblemas menores, sendo capaz de diminuir a complexidade da resolucão. Além disso as características inerentes aos agentes possuem alto potencial para análises mais detalhadas do modelo multidimensional, por exemplo a habilidade social e a proatividade podem permitir a interação com outros modelos de alunos visando comparações e validações do modelo.

Em corroboração ao objetivo do trabalho, a informação do estilo de aprendizagem ao docente pode permitir a correta orientação da sua didática em sala de aula, melhorando qualitativamente o ensino aos seus alunos.

Especificamente, os objetivos deste trabalho são:
\begin{itemize}
 	\item Projetar a arquitetura geral de um SMA por meio de uma metodologia apropriada;
	\item Definir e implementar a arquitetura geral da solução, incluindo os agentes assistentes de cognição, metacognição e afetivo;
	\item Propor uma interface do agente assistente de cognição com os atores externos do sistema: Docente e Estudante;
\end{itemize}

\section{Metodologia}

A metodologia de realização deste trabalho é constituída da revisão bibliográfica, modelagem da arquitetura, implementação e testes em laboratório da solução. A primeira etapa consistiu do estudo dos conceitos de Informática na Educação (IE), Sistemas Multiagentes e levantamentos de uso de SMA em contextos pegagógicos. Este estudo foi importante para a orientação do desenvolvimento embasado na teoria e a comparação das ferramentas existentes com a solução proposta.

Em seguida, foi realizado o levantamento e estudo de metodologias aproriadas para modelagem de SMA existentes para a modelagem e construção da solução. Essas metodologias foram comparadas e uma delas escolhida para a modelagem do SMA.

Em sequência, houve o levantamento do~\emph{framework} necessário para implementação do SMA. Após levantamento e comparação das ferramentas disponíveis, o desenvolvimento da aplicação multiagente foi feito com o~\emph{framework}~\emph{JADE} o qual é completamente desenvolvido na linguagem~\emph{JAVA}, simplificando a implementação de sistemas que respeitam as especificações FIPA. 

Em seguida, foi necessário a escolha de uma ferramenta para desenvolvimento da interface~\emph{web} e a sua integração com o SMA. Baseado nos levantamentos e estudos, o~\emph{framework JBoss Seam} mostrou-se mais viável visto que facilita a construção de aplicações dinâmicas para a internet de forma simples e ágil.

Por fim, foram realizados testes em laboratório da solução proposta. Foram definidos cenários que simulavam o uso da aplicação por meio dos perfis de Aluno e Docente.

\section{Estrutura do Trabalho}
A divisão de capítulos foi feita da seguinte forma:
\begin{itemize}
	\item Capítulo 2: Apresenta uma breve visão sobre as áreas de estudo envolvidas neste trabalho, tais como IE e SMA. Além disso, apresenta algumas ferramentas e tecnologias utilizadas, bem como os trabalhos correlatos e o seu comparativo.
	\item Capítulo 3: Contém a proposta de solução composta pela metodologia, modelagem da arquitetura e implementação.
	\item Capítulo 4: Exibe os testes realizados em laboratório por meio dos perfis dos usuários: Aluno e Docente.
	\item Capítulo 5: Apresenta algumas conclusões e abre perspectivas para trabalhos futuros.
	\item Capítulo 6: Apresenta o apêndice com todos os casos de uso escritos na modelagem do SMA.
\end{itemize}
