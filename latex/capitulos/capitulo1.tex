\chapter{Introdução}

Cada pessoa possui uma forma preferencial de absorção do conhecimento. Seja por imagens, textos, teoria ou prática, durante uma situação de aprendizagem todos tendem a receber e processar melhor as informações que são recebidas de certa maneira, em detrimento a outras. Essa forma de recepção do conhecimento é nomeada estilo de aprendizagem.
 
O acesso à recursos tecnológicos que, outrora caros e difíceis, tornaram-se presentes no cotidiano de muitas pessoas devido a facilidade de aquisição. Dessa forma, a viabilidade do aprendizado por meio do computador aumentou e começa a modificar o paradigma do professor detentor do conhecimento e o único responśavel pela sua transmissão.

Em um ambiente escolar a forma didática escolhida por um docente pode afetar o desempenho dos seus alunos, visto que a forma de transmissão do conhecimento escolhida pode desprivilegiar alguns estudantes. A situação ideal requer a existência de uma personalização do ensino para cada estudante, considerando as características inerentes à sua cognição. Inspirado por esse ideal, algumas ferramentas no âmbito da computação pretendem planejam o ensino personalizado para cada estudante de acordo com o seu perfil de utilização, primeiramente pelo grande acesso da sociedade à computadores e também devido à diversos estudos relacionados na área de educação.

Conhecer os fatores relacionados ao processo de aprendizagem exige que as ferramentas computacionais aplicadas ao ensino consigam determinar eficientemente os estilos de aprendizagem. Assim, a complexidade das aplicações aumenta devido às abordagens que serão empregadas, bem como a exigência de processamento que podem exigir.

Algumas abordagens visam diminuir essa complexidade, permitindo a representação aproximada das características dos alunos em ambientes computacionais, de forma que possam coexistir em um mesmo ambiente vários modelos de estudantes. Por exemplo, o modelo do estudante elaborado na forma multidimensional a partir dos universos cognitivo, metacognitivo e afetivo.

Somado à isso, técnicas de computação distribuída são projetadas para rodar de forma descentralizada, com componentes e serviços rodando em diversos lugares distintos e comunicando-se uma com as outras através de mensagens. A arquitetura dessas aplicações é projetada objetivando o alto paralelismo, exibilidade, interoperabilidade, dentre outros aspectos.

Portanto, determinar o estilo de aprendizagem mostra-se uma estratégia fundamental para que a transmissão de informações e vivências entre alunos e professores torne-se mais eficaz e perceptível, promovendo a criação de informações cada vez mais relevantes para o planejamento, acompanhamento e avaliação dos aprendizes.


\section{Problema}
Os ambientes educacionais de aprendizagem não possuem uma arquitetura apropriada para a inferência de modelos multidimensionais, pois a abordagem baseada em cliente-servidor não é razoável para representação de perfis de alunos no ambiente.

\section{Objetivos}
Tendo em vista o cenário atual apresentado, o presente trabalho tem como objetivo definir uma arquitetura distribuída na Web para auxiliar o processo de ensino-aprendizagem por meio da abordagem de Sistema multiagente (SMA), criando insumos para auxiliar o docente na adoção da melhor estratégia didática de ensino.

Esta abordagem permitirá a construção e manutenção de um modelo multidimensional do estudante, a partir do qual os estilos de aprendizagem desse estudante poderão ser identificados. A abordagem de sistemas multiagentes permite a decomposição do problema na modelagem multidimensional em vários subproblemas menores, sendo capaz de diminuir a complexidade da resolucão. Além disso características inerentes aos agentes, como a habilidade social dos agentes, podem permitir a interação com outros modelos de alunos visando comparações e validações do modelo.

Em corroboração ao objetivo do trabalho, a informação do estilo de aprendizagem ao docente pode permitir a correta orientação da sua didática em sala de aula, melhorando qualitativamente o ensino aos seus alunos.

Especificamente, os objetivos deste trabalho são:
\begin{itemize}
 	\item Projeto da arquitetura geral de um sistema multiagente, utilizando-se de uma metodologia apropriada;
	\item Definir e implementar a arquitetura geral da solução, incluindo os agentes assistentes de cognição, metacognição e afetivo;
	\item Propor uma interface do agente assistente de cognição com os atores externos do sistema: Docente e Estudante;
\end{itemize}

\section{Metodologia}

A metodologia de realização deste trabalho é constituída da revisão bibliográfica, modelagem da arquitetura, implementação e testes em laboratório da solução. A primeira etapa consistiu do estudo dos conceitos de Informática na Educação, Sistemas Multiagentes e levantamentos de uso de SMA em contextos pegagógicos. Este estudo foi importante para a orientação do desenvolvimento embasado na teoria e a comparação das ferramentas existentes com a solução proposta.

Em seguida, foi realizado o levantamento e estudo de metodologias aproriadas para modelagem de SMA existentes para a modelagem e construção da solução. Essas metodologias foram comparadas e uma delas escolhida para o desenvolvimento da aplicação. Após a escolha da metodologia de desenvolvimento, foi feita a modelagem do SMA com a metodologia em questão.

O desenvolvimento da aplicação multiagente foi feito com o~\emph{framework}~\emph{JADE}. Que é completamente desenvolvido na linguagem~\emph{JAVA} e simplifica a implementação de Sistemas Multiagentes (SMA) que cumprem as especificações FIPA. 

Em seguida, foi necessário a escolha de uma ferramenta para desenvolvimento da interface web e a sua integração com o sistema multiagente. Baseado nos estudos feitos, o~\emph{framework JBoss Seam} mostrou-se mais viável visto que facilita a construção de aplicações dinâmicas para a internet de forma simples e ágil.

Por fim, foram realizados testes em laboratório da solução proposta. Foram definidos cenários que simulavam o uso da aplicação por meio dos perfis de Aluno e Docente.

\section{Estrutura do Trabalho}
A divisão de capítulos foi feita da seguinte forma:
\begin{itemize}
	\item Capítulo 2: Apresenta uma breve visão sobre as áreas de estudo envolvidas neste trabalho: Informática na Educação, Sistema Multiagente. Além disso, apresenta algumas ferramentas e tecnologias utilizadas, bem como os trabalhos correlatos e um comparativo.
	\item Capítulo 3: Contém a proposta de solução composta pela metodologia, modelagem da arquitetura e implementação.
	\item Capítulo 4: Exibe os testes realizados em laboratório por meio dos perfis dos usuários: Aluno e Docente.
	\item Capítulo 5: Apresenta algumas conclusões e abre perspectivas para trabalhos futuros.
\end{itemize}
