\chapter{Introdução}

Cada pessoa possui uma forma preferencial de absorção do conhecimento. Seja por imagens, textos, teoria ou prática, durante uma situação de aprendizagem todos tendem a receber e processar melhor as informações que são recebidas de certa maneira, em detrimento a outras. Essa forma de recepção do conhecimento é nomeada estilo de aprendizagem.
 
O acesso à recursos tecnológicos que, outrora caros e difíceis, tornaram-se presentes no cotidiano de muitas pessoas devido a facilidade de aquisição. Dessa forma, a viabilidade do aprendizado por meio do computador aumentou e começa a modificar o paradigma do professor detentor do conhecimento e o único responśavel pela sua transmissão.

Em um ambiente escolar a forma didática escolhida por um docente pode afetar o desempenho dos seus alunos, visto que a forma de transmissão do conhecimento escolhida pode desprivilegiar alguns estudantes. A situação ideal requer a existência de uma personalização do ensino para cada estudante, considerando as características inerentes à sua cognição. Inspirado por esse ideal, algumas ferramentas no âmbito da computação pretendem planejam o ensino personalizado para cada estudante de acordo com o seu perfil de utilização, primeiramente pelo grande acesso da sociedade à computadores e também devido à diversos estudos relacionados na área de educação.

Conhecer os fatores relacionados ao processo de aprendizagem exige que as ferramentas computacionais aplicadas ao ensino consigam determinar eficientemente os estilos de aprendizagem. Assim, a complexidade das aplicações aumenta devido às abordagens que serão empregadas, bem como a exigência de processamento que podem exigir.

Algumas abordagens visam diminuir essa complexidade, permitindo a representação aproximada das características dos alunos em ambientes computacionais, de forma que possam coexistir em um mesmo ambiente vários modelos de estudantes. Por exemplo, o modelo do estudante elaborado na forma multidimensional a partir dos universos cognitivo, metacognitivo e afetivo.

Somado à isso, técnicas de computação distribuída são projetadas para rodar de forma descentralizada, com componentes e serviços rodando em diversos lugares distintos e comunicando-se uma com as outras através de mensagens. A arquitetura dessas aplicações é projetada objetivando o alto paralelismo, exibilidade, interoperabilidade, dentre outros aspectos.

Portanto, determinar o estilo de aprendizagem mostra-se uma estratégia fundamental para que a transmissão de informações e vivências entre alunos e professores torne-se mais eficaz e perceptível, promovendo a criação de informações cada vez mais relevantes para o planejamento, acompanhamento e avaliação dos aprendizes.


\section{Problema}
Os ambientes educacionais de aprendizagem não possuem uma arquitetura apropriada para a inferência de modelos multidimensionais, pois a abordagem baseada em cliente-servidor não é razoável para representação de perfis de alunos no ambiente.

\section{Objetivos}
Tendo em vista o cenário atual apresentado, o presente trabalho tem como objetivo definir uma arquitetura distribuída na Web para auxiliar o processo de ensino-aprendizagem por meio da abordagem de Sistema multiagente (SMA).

Esta abordagem permitirá a construção e manutenção de um modelo multidimensional do estudante, a partir do qual os estilos de aprendizagem desse estudante poderão ser identificados e informados ao docente. A abordagem de sistemas multiagentes permite a decomposição do problema na modelagem multidimensional em vários subproblemas menores, diminuindo a complexidade da resolucão. Além disso, a habilidade social dos agentes pode permitir a interação com outros modelos de alunos visando comparações e validações do modelo.

Especificamente, os objetivos específicos deste trabalho são:
\begin{itemize}
 	\item Projeto da arquitetura geral de um sistema multiagente, utilizando-se de uma metodologia apropriada;
	\item Definir e implementar a arquitetura geral da solução, incluindo os agentes assistentes de cognição, metacognição e afetivo;
	\item Propor uma interface do agente assistente de cognição com os atores externos do sistema: Docente e Estudante;
\end{itemize}

\section{Metodologia}
A metodologia utilizada para a realização deste trabalho é composta das seguintes atividades:
\begin{itemize}
 	\item Estudo dos conceitos de Informática na Educação, Sistemas Multiagentes e metodologia MASE, importantes para a orientação do desenvolvimento deste trabalho com base na teoria e garantir as melhores práticas.
	\item Estudo e pesquisa do~\emph{middleware} JADE e a sua integração com aplicações externas.
	\item Baseado nos estudos feitos, desenvolver a modelagem da Solução utilizando a metodologia MASE a qual define o desenvolvimento da solução através de uma série de diagramas que irão justificar as escolhas da arquitetura proposta.
	\item Desenvolvimento da aplicação multiagente com base na pesquisa a respeito do~\emph{JADE}.
	\item Desenvolvimento da aplicação web e a sua integração com o sistema multiagente.
	\item Testes em laboratório da solução desenvolvida e suas conclusões. Os testes deste trabalho serão realizados por meio de simulações com alunos e docentes. Considerando o cenário do aluno, ele deve autenticar-se e utilizar o sistema para a verificação do seu estilo de aprendizagem, bem como visualizar a criação dos agentes do seu grupo de trabalho. O cenário do docente deve possibilitar a visualização do estilo de aprendizagem dos seus alunos.
\end{itemize}

Usando tecnologias existentes e consolidadas, a arquitetura proposta irá usar o framework~\emph{JADE}, que é completamente desenvolvido na linguagem~\emph{JAVA} e simplifica a implementação de Sistemas Multiagentes (SMA) que cumprem as especificações FIPA. A arquitetura proposta também englobará uma interface web que utiliza a plataforma~\emph{open source JBoss Seam}, desenvolvida para auxiliar a construção de aplicações dinâmicas para a internet de forma simples e ágil.

\section{Estrutura do Trabalho}
Este trabalho consiste na divisão de capítulos visando facilitar a leitura e organizar os conceitos que perfazem o desenvolvimento deste trabalho:
\begin{itemize}
	\item Capítulo 2 apresenta uma reve visão sobre as áreas de estudo envolvidas neste trabalho que são Informática na Educação, Sistema Multiagente, bem como algumas ferramentas e tecnologias.
	\item Capítulo 3 contém a proposta de solução composta pela metodologia, modelagem da arquitetura e implementação.
	\item Capítulo 4 exibe os testes realizados em laboratório por meio dos cenários de uso definidos: Aluno e Docente.
	\item Por fim, o capítulo 5 apresenta as conclusões e abre perspectivas para trabalhos futuros.
\end{itemize}
