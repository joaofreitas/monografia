\begin{usecase}

\addtitle{Caso de Uso 1}{Determinar Modelagem Cognitiva do Aluno}

\addfield{Escopo:}{Sistema e Ambiente(s) Virtuais(ns) de Aprendizagem(s)}
\addfield{Nível:}{Objetivo do sistema}
\addfield{Ator Principal:}{Sistema}

\additemizedfield{Interessados e Interesses:}{
	\item Sistema: Manter um modelo do estilo de aprendizagem do estudante a partir dos dados gerados do formulário, bem como com algum Ambiente Virtual de Aprendizagem.
	\item Aluno: Preencher o questionário de estilos de aprendizagem e interagir com um AVA.
}

\addfield{Pré-condições:}{Não se aplica}
\addfield{Pós-condições:}{}

\addscenario{Cenário de Sucesso Principal:}{
	\item O Aluno faz o primeiro login no sistema.
	\item O Sistema reconhece o perfil do aluno e cria um Grupo de Trabalho (GT) de agentes.
	\item O Aluno deverá preencher o questionário de estilos de aprendizagem.
	\item O Sistema avalia as respostas preenchidas pelo aluno e, conforme a literatura de psicologia, determinará o estilo de aprendizagem por meio da avaliação explícita.
	\item O Sistema construirá o primeiro modelo do aluno, com base no seu estilo de aprendizagem inicial.
	\item O Sistema salva os dados, a fim de notificar o docente dos dados obtidos.
	\item O Sistema notifica o Aluno sobre o seu estilo de aprendizagem.
	\item O Sistema exibe a lista de AVA disponíveis para o Aluno.
}

\addscenario{Extensões:}{
	\item[1.a] Falha no login:
		\begin{enumerate}
		\item[1.] O Sistema mostra mensagem de falha.
		\item[2.] O Docente retorna ao passo 1.
		\end{enumerate}
	\item[2.a] Inferência implícita de estilo de aprendizagem:
		\begin{enumerate}
		\item[1.] O Aluno faz login no sistema
		\item[2.] O Sistema reconhece o perfil do aluno e cria um grupo de trabalho de agentes.
		\item[3.] O Sistema exibe a lista de AVA disponíveis para o Aluno.
		\item[4.] O Aluno seleciona o AVA
		\item[5.] O Sistema é notificado do AVA escolhido
		\item[6.] O AVA registra as ações do Aluno e envia para o Sistema.
		\item[7.] O AVA interage com o Sistema por meio do caso de uso.
		\item[8.] O Sistema utiliza os resultados de atividades feitas pelo aluno para inferir o seu estilo de aprendizagem, levando em consideração a literatura da psicologia relativa à atividade desenvolvida.
		\item[9.] O Sistema compara então os dados obtidos com os dados previamente armazenados, a fim de obter um grau de certeza do estilo de aprendizagem que foi construído.
		\item[10.] O Sistema salva os dados, a fim de notificar o Docente dos dados obtidos.
		\end{enumerate}
	\item[3.a] Inferência implícita de estilo de aprendizagem:
		\begin{enumerate}
		\item[1.] O Sistema analisa os dados obtidos do ambiente para determinar o desempenho do Aluno na atividade, levando em consideração a literatura pedagógica relativa.
		\item[2.] O Sistema salva os dados no banco de dados, a fim de notificar o Docente dos dados obtidos.
		\end{enumerate}
}

\addscenario{Requisitos Especiais:}{
	\item Não se aplica
}

\addscenario{Lista de Variantes Tecnologias de Dados:}{
	\item Não se aplica
}

\addfield{Frequência de Ocorrência:}{Sempre}

\end{usecase}
