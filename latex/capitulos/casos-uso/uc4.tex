\begin{usecase}

\addtitle{Caso de Uso 4}{Determinar Modelagem Metacognitiva do Aluno} 

\addfield{Escopo:}{Sistema e Ambiente(s) Virtuais(ns) de Aprendizagem(s)}
\addfield{Nível:}{Objetivo do sistema}
\addfield{Ator Principal:}{Sistema}

\additemizedfield{Interessados e Interesses:}{
	\item Sistema: Criar e manter um modelo metacognitivo do estudante a partir dos dados gerados da sua interação com um Ambiente Virtual de Aprendizagem (AVA).
	\item Aluno: Visualizar o modelo afetivo determinado pelo Sistema.
}

\addfield{Pré-condições:}{}
\addfield{Pós-condições:}{}

\addscenario{Cenário de Sucesso Principal:}{
	\item O Aluno faz login no Sistema.
	\item O Sistema exibe a lista de AVAs disponíveis para o Aluno.
	\item O Aluno seleciona o AVA.
	\item O Sistema é notificado do AVA escolhido
	\item O AVA registra as ações do Aluno e envia para o Sistema.
	\item O ambiente interage com o Sistema por meio do caso de uso UC5
	\item O Sistema utiliza os resultados de atividades feitas pelo aluno para criar um modelo metacognitivo do estudante, levando em consideração a literatura da psicologia relativa à atividade desenvolvida.
	\item O Sistema compara então os dados obtidos com os dados previamente armazenados no banco de dados, caso existam, a fim de atualizar o modelo previamente criado.
	\item O Sistema salva os dados no banco de dados, a fim de notificar o Docente dos dados obtidos.
}

\addscenario{Extensões:}{
	\item[1.a] Falha no login:
		\begin{enumerate}
		\item[1.] O Sistema mostra mensagem de falha.
		\item[2.] O Docente retorna ao passo 1.
		\end{enumerate}
}

\addscenario{Requisitos Especiais:}{
	\item Não se aplica
}

\addscenario{Lista de Variantes Tecnologias de Dados:}{
	\item Não se aplica
}

\addfield{Frequência de Ocorrência:}{Sempre}

\end{usecase}
