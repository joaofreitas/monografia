\chapter{Fundamentos básicos}

Este capitulo apresenta os principais conceitos e definições necessários para o entendimento deste trabalho. A seção 2.1 apresenta definições de sistemas multiagentes.
A seção 2.2 apresenta uma breve explicação sobre a arquitetura da plataforma JAMA, bem como o seu funcionamento.
A seção 2.3 contém uma breve introdução sobre dependabilidade.
Por fim, a seção 2.4 tem um texto expondo o funcionamento do JAMA, bem como a sua arquitetura.

\section{Sistemas Multiagentes}

\subsection{Inteligência artificial}

Antes de explicarmos o conceito de sistemas multiagentes (SMA), é necessário mostrar conceitos que são base para o entendimento de SMA. Inicia-se apresentando alguns conceitos de Inteligencia Artificial (IA). De acordo com~\cite{poole98} identificamos que a definição de IA pode variar em duas dimensões principais. Usando a definição de sistemas computacionais que agem racionalmente temos:

\begin{quote}
\emph{Computational Intelligence is the study of the design of intelligent agents.}
\end{quote}

Nessa definição, é importante ressaltar que o agente é uma entidade que atua racionalmente, esperando-se que essa racionalidade e outras características o diferencie de simples programas.

Com o crescimento dos estudos relacionado a este campo, a inteligência artificial ganhou várias áreas de atuação e resolução de problemas no nosso cotidiano. Um dos problemas é a necessidade de executar aplicações que resolvem problemas de alta complexidade. Essas aplicações podem exigir um hardware muito caro para a execução, ou então pode-se usar a abordagem de distribuí-la em vários computadores que dividem a sua execução. É justamente onde entra a inteligência artificial distribuída: São sistemas que são compostos por vários agentes coletivos, ou seja, distribuem o trabalho uns com os outros. Cada agente pode possuir uma capacidade diferente, sendo possível realizar a tarefa de modo paralelo. 

\subsection{Agente}

De acordo com~\cite{novig95}, agentes são entidades (reais ou virtuais) que funcionam de forma autônoma em um ambiente, ou seja, não necessitam de intervenção humana para realizar processamento. Esse ambiente de funcionamento do agente geralmente contém vários outros agentes e é possível a comunicação entre eles através do ambiente por meio de troca de mensagens.

Em geral o funcionamento de agentes acontece de forma a perceberem o ambiente em que estão por meio de sensores, fazem análises com base nessa interação inicial e por fim podem agir sobre o ambiente de forma a modifica-lo por meio de efetuadores. A figura~\ref{fig:agente-basico} apresenta um resumo do que foi dito.

\begin{figure}
	\centering
	\includegraphics[scale=0.75]{images/agente-basico.png}
	\caption{Esquematização do funcionamento básico de um agente em um ambiente.}
	\label{fig:agente-basico}
\end{figure}

Agentes racionais seguem o princípio de racionalidade básico: sempre objetivam suas ações pela escolha da melhor ação possível segundo seus conhecimentos. Logo é possível inferir que a ação de um agente nem sempre alcança o máximo desempenho, sendo desempenho o parâmetro definido para medir o grau de sucesso da ação de um agente com base nos seus objetivos.

Como dito anteriormente, agentes estão presentes em um ambiente. O agente não tem controle total do ambiente, ele pode no máximo influenciá-lo com a sua atuação. Podemos separar ambientes em classes: Software, Físico e Relidade virtual (simulação de ambientes reais em software). De acordo com~\cite{wooldridge04} temos, em geral, ambientes tem propriedades inerentes que dizem respeito ao seu funcionamento:

\begin{itemize}
	\item Observável: Neste tipo de ambiente, os sensores dos agentes conseguem ter percepção completa do ambiente. Por exemplo, um sensor de movimento consegue ter visão total em um ambiente aberto.
	\item Determinística: O próximo estado do ambiente é sempre conhecido dado o estado atual do ambiente e as ações dos agentes. O oposto do ambiente determinístico é o estocástico, quando não temos certeza do estado do ambiente. Por exemplo, agentes dependentes de eventos climáticos.
	\item Episódico: A experiência do agente é dividida em episódios, onde cada episódio é a percepção do agente e a sua ação.
	\item Sequêncial: A ação tomada pelo agente pode afetar o estado do ambiente e ocasionar na mudança de estado
	\item Estático: O ambiente não é alterado enquanto um agente escolhe uma ação.
	\item Discreto: Existe um número definido de ações e percepções do agente para o ambiente em cada turno.
	\item Contínuo: As percepções e ações de um agente modificam-se em um espectro contínuo de valores. Por exemplo, temperatura de um sensor muda de forma contínua.
\end{itemize}

Na tabela~\ref{lista_agentes} mostramos alguns exemplos de agentes, apresentando as suas características já discutidas nesse trabalho.

\begin{table}
	\caption{Listagem de sistemas multiagentes com propriedades de medida de performance, ambiente, atuadores e sensores}
	\begin{tabular}{|p{3cm} | p{3cm} | p{2cm}| p{3cm} | p{3cm} |}
		\hline
		\textbf{Tipo de agente}	& \textbf{Medida de performance} & \textbf{Ambiente} & \textbf{Atuadores}  & \textbf{Sensores}	\\
		\hline
		Sensores de estacionamento	& Avarias no veículo & Carro e garagens & Freio do carro, controle de velocidade & Sensor de proximidade	\\
		\hline
		Jogos com oponente computador	& Quantidade de vitórias &	Software & Realizar jogada & Percepção do tabuleiro	\\
		\hline
		Agentes hospitalares		& Saúde do paciente & Paciente, ambiente médico & Diagnósticos & Entrada de sintomas do paciente	\\
		\hline
	\end{tabular}
	\label{lista_agentes}
\end{table}
 
A primeira linha da tabela~\ref{lista_agentes} é apresentado um exemplo de um agente atuando em um veículo como um sensor de estacionamento. Responsável por auxiliar o motorista no ato de estacionar o carro, o seu ambiente é da classe físico (considerando o carro e o ambiente onde está o carro). Seu sensor de proximidade é a percepção do ambiente e caso detecte que está próximo de um obstáculo pode atuar nos freios dos carros diminuindo a velocidade e evitando colisões. Avarias no carro podem indicar um mal funcionamento do sensor.

A segunda linha da tabela é apresentado u exemplo de agente atuando em um jogo qualquer. Esse ambiente é dito dinâmico, pois a cada jogada de um oponente (real ou não), o agente irá analisar a jogada feita pelo seu oponente, irá calcular sua próxima jogada e irá realizá-la. O objetivo principal do agente é a vitória. O ambiente que o agente atua é um software e o seu atuador é um algum mecanismo que permite que ele realize a jogada. O sensor é o mecanismo no qual o agente irá perceber a jogada realizada pelo oponente.

Por fim, última linha da tabela~\ref{lista_agentes} expõe um exemplo de um agente médico atuando em um ambiente estático um paciente. Esse ambiente é dito estático por que não será alterado pelo agente nesse exemplo, mas podendo ser diferente dependendo da aplicação. O objetivo principal é monitorar a saúde do paciênte, logo a medida de performance será a aproximação ou não do diagnóstico médico. Seu atuador não será diretamente no ambiente (corpo humano), será na forma de relatórios médicos e seus sensores podem variar de acordo com a doença a ser monitorada.

Conforme podemos encontrar em~\cite{wooldridge04}, podemos definir algumas noções gerais de agentes. A primeira, chamada de noção fraca, contém a maior parte dos agentes. Ela compreende os aspectos de \emph{reatividade}, \emph{proatividade} e \emph{habilidade social}. O conceito de reatividade  está ligado com o agente perceber o ambiente e reagir. Proatividade é a característica do agente tomar a iniciativa e agir sem a necessidade de nenhum estímulo. Habilidade social é a capacidade de interação com outros agentes.

Já a noção forte de agente envolve os seguintes aspectos: , veracidade, benevolência
\begin{itemize}
	\item Mobilidade: O Agente deve pode mover-se no ambiente, por exemplo, em uma rede.
	\item Veracidade: Agente não comunica informações falsas.
	\item Benevolência: Agente ajudará os outros.
	\item Racionalidade: O agente não irá agir de forma a impedir a realização de seus objetivos.
	\item Cooperação: O agente coopera com o usuário.
\end{itemize}

\subsection{Arquitetura de agentes}

A arquitetura de agentes varia de acordo com a complexidade da sua autonomia, ou seja, com a capacidade de reagir aos estímulos do ambiente. Conforme verificado no livro de ~\cite{novig95}, os tipos de arquitetura são: orientadas à tabela, reflexiva simples, reflexiva baseado em modelo, baseada em objetivo, baseada em utilidade.

A primeira arquitetura a ser explorada é o agente orientado à tabelas. Todas as ações dos agentes dessa arquitetura são conhecidas e estão gravadas em uma tabela. Assim, quando o agente receber o estímulo ele já terá a ação a ser tomada previamente gravada em sua memória. Logo para construir esse tipo de agente, fica claro que além de saber todas percepções possívels, será necessário definir ações apropriadas para todas. Isso levará a tabelas muito complexas e o tamanho pode facilmente passar a ordem de milhões dependendo do número de entradas.

A arquitetura reflexiva simples é um dos tipos mais simples de agente. Nele, o agente seleciona a ação com base unicamente na percepção atual, desconsiderando assim uma grande tabela de decisões. A decisão é tomada com base de regras condição-ação: Se uma condição ocorrer, uma ação será tomada. Por exemplo, vamos supor um agente médico que determina o diagnóstico de uma doença no paciente caso exista alguma anomalia no organismo (Por exemplo, paciente com febre). Uma condição-ação poderia ser:

if anomalia-organismo then diagnóstico-médico

Esse tipo de agente é bastante simples, o que é uma vantagem comparado à arquitetura de tabela. Porém, essa abordagem requer um ambiente totalmente observável, visto que esse tipo de agente possui uma inteligência bastante limitada. No exemplo do agente médico existem diversas maneiras de se detectar uma anomalia no organismo do paciente, seria necessário conhecer todas as formas para usarmos uma abordagem reativa simples.

A arquitetura reflexiva baseada em modelos funciona de maneira similar a anterior. Nessa abordagem, é levado em conta a parte do ambiente que não é visível neste momento. E para saber o ``momento atual'' de um agente, é necessário guardar a informação de estado consigo. Para atualizar o estado do agente, é necessário conhecer como o mundo desenvolve-se independente do agente (no caso do exemplo, como o organismo funciona) e é necessário saber as ações dos agentes no ambiente. Esses dois conhecimentos do ambiente são chamados de \textbf{modelo do mundo}. O agente que usa esse tipo de abordagem é chamado de agente baseado em modelo.

Na arquitetura reflexiva baseada em objetivo, as ações do agente são tomadas apenas se o aproximam de alcançar um objetivo. Para isso, será necessário algo além do estado atual do ambiente: Será necessário informações do objetivo a ser atingido. Assim o agente pode combinar as informações do estado e o objetivo para determinar se deve ou não agir sobre o ambiente. Essa arquitetura porém é obviamente mais complexa e de certa forma ineficiente. Porém ela permite uma maior flexibilização das ações em determinados ambientes, visto que suas decisões são representadas de forma explícita e podem ser modificadas. É interessante notar que esse tipo de arquitetura não trata ações com objetivos conflitantes.

E por fim, a arquitetura reflexiva baseada em utilidade não utiliza apenas objetivos para realizar a próxima decisão, mas dá ao agente a capacidade de fazer comparações sobre o estado do ambiente e as ações a serem tomadas: Quais delas são mais baratas, confiáveis, baratas, rápidas do que as outras. A capacidade de avaliação do agente chama-se função de utilidade, que mapeia uma sequência de estados em um número real que determina o grau de utilidade. Esse mecanismo possibilita a decisão racional de escolha entre vários objetivos conflitantes. Por exemplo, escolher entre um objetivo mais barato ao invés de escolher entre o mais rápido.

\subsection{Sistemas Multiagentes}

Sistemas multiagentes são sistemas compostos por vários agentes capazes de se comunicar, possuindo uma linguagem de alto nível para isso. O agente deve ser conhecimento para realizar uma determinada tarefa e pode ou não cooperar com outros agentes para realizá-la.

Fica claro nessa definição que sistemas multiagentes

De acordo com~\cite{sarmento11}, podemos encontrar as seguintes características principais de ambientes em SMAs:
\begin{itemize}
	\item Ambientes SMAs fornecem protocolos específicos para comunicação e interação. Cada ambiente tem as suas particularidades: Alguns são em uma única máquina, outros são compartilhados com o mundo real e outros são distribuídos. Cabe a cada ambiente definir um protocolo onde todos agentes devem obedecer para comunicar-se.
	\item SMAs são tipicamente abertos.
	\item SMAs contém agentes que são autônomos e individualistas.
\end{itemize}



























