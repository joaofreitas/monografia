\chapter{Conclusões e Trabalhos Futuros}
Este capítulo analisa o trabalho realizado discutindo o alcance dos objetivos por meio da arquitetura e protótipo desenvolvidos. Em seguida, é indicado uma série de possíveis trabalhos futuros os quais podem ser desenvolvidos a partir da plataforma Frank.

A arquitetura proposta objetivou traçar os modelos cognitivo, afetivo e metacognitivo do aluno, a fim de criar insumos para auxiliar o docente na adoção da melhor estratégia didática de ensino.

Após o estudo, definição e implementação do SMA Frank foi possível obter a arquitetura do sistema e o protótipo com uma plataforma~\emph{web} capaz de interagir tanto com os alunos, quanto com os seus professores, além da implementação dos seguintes agentes:
\begin{itemize}
 	\item Dois agentes de interfaces com o ambiente externo;
	\item Um agente para controle do ambiente do SMA;
	\item Um grupo de trabalho para cada aluno, composto por 4 agentes: O grupo de trabalho e os agentes cognitvo, metacognitivo e afetivo.
\end{itemize}

Através dos resultados alcançados com o estudo, definição e implementação da plataforma, conclui-se que este trabalho pode contribuir para a área de IE, através de uma plataforma de ensino-aprendizagem que visa auxiliar a didática de ensino do docente ao perceber a melhor forma que seus estudantes podem aprender. Consequentemente este trabalho auxilia também os alunos, devido a possibilidade de melhoria na qualidade de ensino aumentar o desempenho do aprendizado. 

A arquitetura da solução está proposta, porém é necessário a continuação de diversos aspectos da aplicação. Devido à imersão nos campos de IE e SMA, este trabalho possui um potencial considerável em relação aos trabalhos futuros.

Primeiramente, como trabalho futuro faz-se necessário modelar as ontologias que podem definir os universos metacognitivo e afetivo. Em seguida, definir também como podem ser feitas as inferências explícitas destes modelos, completando assim a modelagem multidimensional do aluno.

A partir da definição destes modelos, será possível estudar uma forma de aprofundar algumas características dos agentes. Por exemplo, estudar e trabalhar mais a habilidade social dos agentes do grupo de trabalho, de forma que possam compartilhar conhecimentos prévios para a inferência de modelos de outros alunos. Ou então, a partir dos modelos afetivos e metacognitivo, como os agentes podem atingir seus objetivos mais rapidamente ao inferir estes modelos.

Outro ponto importante é a necessidade de integrar a aplicação com vários AVA. Primeiramente, faz-se necessário um levantamento inicial de possibilidades de ambientes os quais desejam-se integrar com o Frank. Deve-se também determinar o domínio do conhecimento destes ambientes, além de inferir o modelo do estudante. A partir dos dados obtidos com a interação do aluno com o AVA, implementar a forma de inferência nos agentes: cognitivo, metacognitivo e afetivo.

A fim de validar a arquitetura proposta, realizar testes e simulações reais com alunos e docentes em um ambiente escolar é outro aspecto primordial. Para tanto, deve-se escolher e visitar ambientes escolares. Em seguida, justificar e determinar quais alunos participarão da validação. A validação do cenário do docente faz-se necessária também e deve-se estudar como as informações disponibilizadas pelo sistema podem auxiliá-lo. Uma sugestão de projeto é o desenvolvimento de métodos e estratégias de aprendizagem~\cite{muhlbeier12} com base nos indicadores gerados, propondo atividades, conteúdos e ferramentas ao docente de acordo com o estilo de aprendizagem dos alunos.

Por fim, é desejável que a plataforma~\emph{web} possua um controle maior do ambiente SMA. Ou seja, uma administração~\emph{web} que controle toda a aplicação, sendo possível a criação e destruição remota de agentes, o balanceamento dos agentes entre as diversas máquinas disponibilizadas para aplicação, a visualização do estado atual de cada grupo de trabalho de cada aluno, dentre outras possibilidades.
