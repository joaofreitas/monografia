\chapter{Conclusões e Trabalhos Futuros}
Este capítulo analisa o trabalho realizado discutindo o alcance dos objetivos por meio da arquitetura e protótipo desenvolvidos. Em seguida, são indicados uma série de possíveis trabalhos futuros em cima da plataforma Frank.

A arquitetura proposta objetivou traçar os modelos cognitivo, afetivo e metacognitivo do aluno, criando insumos para auxiliar o docente na adoção da melhor estratégia didática de ensino.

Ao fim deste trabalho, foi possível definir e implementar a arquitetura geral do SMA Frank:
\begin{itemize}
 	\item Dois agentes de interfaces com o ambiente externo;
	\item Um agente para controle do ambiente do SMA;
	\item Um grupo de trabalho para cada aluno, composto por 4 agentes: O grupo de trabalho e os agentes cognitvo, metacognitivo e afetivo.
\end{itemize}

Além disso, a implementação do protótipo da plataforma web é capaz de interagir tanto com os alunos, quanto com os seus professores.

Para tanto, com os objetivos alcançados, concluí-se que este trabalho possui uma potêncial importância para a sociedade como um todo, pois auxilia o complexo e longo processo de aprendizagem de alunos e possui capacidade para a melhoria significativa do trabalho do docente.

A arquitetura do SMA Frank está proposta, porém é necessária a continuação de diversos aspectos na arquitetura. Primeiramente, como trabalho futuro faz-se necessário obter a modelagem metacognitiva e afetiva dos alunos, além das suas respectivas formas de inferências implícitas. Faz-se necessário também a modelagem de ontologias para definir o universo dos modelos multidimensionais.

A partir da definição destes modelos, é possível estrutura melhor a racionalidade dos agentes no sentido de...

É necessário uma integração com um Ambiente Virtual de Aprendizagem e realizar testes reais com alunos e docentes para a validação das inferências realizadas.

Por fim, é desejável que a plataforma web possua um controle maior do ambiente SMA. Ou seja, uma administração web que controle todo o SMA, sendo possível a criação e remoção remota de agentes, o balanceamento dos agentes entre as diversas máquinas disponibilizadas para aplicação, a visualização do estado atual de cada grupo de trabalho de cada aluno.
