\chapter{Conclusões e Trabalhos Futuros}
Este capítulo analisa o trabalho realizado discutindo o alcance dos objetivos por meio da arquitetura e protótipo desenvolvidos. Em seguida, são indicados uma série de possíveis trabalhos futuros em cima da plataforma Frank.

A arquitetura proposta objetivou traçar os modelos cognitivo, afetivo e metacognitivo do aluno, criando insumos para auxiliar o docente na adoção da melhor estratégia didática de ensino.

Ao fim deste trabalho, foi possível definir e implementar a arquitetura geral do SMA Frank:
\begin{itemize}
 	\item Dois agentes de interfaces com o ambiente externo;
	\item Um agente para controle do ambiente do SMA;
	\item Um grupo de trabalho para cada aluno, composto por 4 agentes: O grupo de trabalho e os agentes cognitvo, metacognitivo e afetivo.
\end{itemize}

Além disso, a implementação do protótipo da plataforma web é capaz de interagir tanto com os alunos, quanto com os seus professores.

Para tanto, com os objetivos alcançados, concluí-se que este trabalho possui uma potêncial importância para a sociedade como um todo, pois auxilia o complexo e longo processo de aprendizagem de alunos e possui capacidade para a melhoria significativa do trabalho do docente.

A arquitetura da solução está proposta, porém é necessário a continuação de diversos aspectos da aplicação. Devido a natureza deste trabalho, a possibilidade de trabalhos futuros é imensa.

Primeiramente, como trabalho futuro faz-se necessário modelar as ontologias que podem definir os universos metacognitivo e afetivo. Em seguida, definir também como podem ser feitas as inferências explícitas destes modelos, completando assim a modelagem multidimensional do aluno.

A partir da definição destes modelos, é possível estudar a forma de aprofundar algumas características dos agentes. Por exemplo, estudar e trabalhar mais a habilidade social dos agentes do grupo de trabalho, de forma que possam compartilhar conhecimentos prévios para a inferência de modelos de outros alunos. Ou então, a partir dos modelos afetivos e metacognitivo, como os agentes podem atingir mais rapidamente seu objetivo durante à inferência destes universos.

Um outro ponto importante é a necessidade de integrar a aplicação com vários Ambientes Virtuais de Aprendizagem. Primeiramente, faz-se necessário um levantamento inicial de possibilidades de Ambientes. Em seguida, deve-se determinar o domínio do conhecimento destes Ambientes os quais desejam-se inferir o modelo do estudante. Por fim, a partir dos dados obtidos com a interação do aluno com o AVA, implementar a forma de inferência nos agentes cognitivo, metacognitivo e afetivo.

A fim de validar a arquitetura proposta, realizar testes e simulações reais com alunos e docentes em um ambiente escolar é outro aspecto primordial. Para tanto, deve-se escolher e visitar ambientes escolares. Em seguida, justificar e determinar quais alunos participarão da validação. Por fim, é necessário validar o cenário do docente, verificando a forma como as informações disponibilizadas pelo sistema podem auxiliá-lo. Uma sugestão de projeto é o desenvolvimento de métodos e estratégias de aprendizagem~\cite{muhlbeier12} com base nos indicadores gerados, propondo atividades, conteúdos e ferramentas ao docente de acordo com o estilo de aprendizagem dos alunos.

Por fim, é desejável que a plataforma web possua um controle maior do ambiente SMA. Ou seja, uma administração web que controle toda a aplicação, sendo possível a criação e destruição remota de agentes, o balanceamento dos agentes entre as diversas máquinas disponibilizadas para aplicação, a visualização do estado atual de cada grupo de trabalho de cada aluno, dentre outras possibilidades.
